\documentclass[10.5pt]{book}

\usepackage{verse}
\usepackage{tocloft}

\usepackage[utf8]{inputenc}
%\usepackage[czech]{babel}
\PassOptionsToPackage{bookmarks, colorlinks=false, hidelinks}{hyperref}
% Use PoetryTeX; http://www.ctan.org/pkg/poetrytex
\usepackage[numberpoems, clearpageafterpoem, useincipits]{poetrytex}

% Use the PA5 paper size
\usepackage[paperwidth=140mm,paperheight=210mm]{geometry}

\renewcommand{\pttitle}{Zpěvník}
\renewcommand{\ptsubtitle}{hudop.cz}
\renewcommand{\ptauthor}{Óňa}
\renewcommand{\ptdate}{2017/07/30}
\renewcommand{\ptdedication}{Made by HUDOP:\\*
This is dedicated to\\*
Someone else, not you.}
\renewcommand{\contentsname}{Osnova}


\begin{document}

\maketitle
\makededication

% Number pages with small roman numerals (i, ii, iii, iv...)
% \frontmatter
% Start numbering pages with normal arabic numerals.
\mainmatter

% TOC %
\maketoc

% TOP %
\renewcommand*{\topname}{Songlist} % Name for the table of songs
\maketop

\section{České}
Předmluva k českým songům.

\newpage
\thispagestyle{empty}

% Start numbering pages with normal arabic numerals.
% \mainmatter

\begin{poem}{Za pokladem Černé perly}{Hurá do přírody}

\settowidth{\versewidth}{"Hodokvas vítězný - hřeš, jak jen chceš."}

Na bárce z krunýře pějem svůj verš,\\*
jak se tak ploužíme líně jak veš.\\*
Velrybu bílou znal pražský král,\\*
pro něj ji chytit nám za úkol dal.\\*
"Hodokvas vítězný - hřeš, jak jen chceš."

Zemřela chýra - nás přepad' žal, \\*
tělo nechť plaví proud po vlnách v dál.\\*
Výprava její se skončila v půli,\\*
toť dílo krysí a krysám jen k vůli,\\*
přísahu porušil ten sprostý král.

Do moře hladem pad' nejeden pták,\\*
shůry až do vln, jak stáh' by ho hák.\\*
Keblovský hlídkař se svědectvím černým,\\*
jenž v plášti hlídkuje nad mořem věrným,\\*
došel až za známé hranice map.

Na bárce z krunýře přes moře jdem, \\*
vstříc krajům mlžným, kde skrytá je zem.\\*
Nespočet na moři spatříme rán,\\*
smát se i plakat zří nás oceán,\\*
dřív, nežli vstoupíme na rodnou zem,\\*
dřív, nežli vstoupíme na naši rodnou zem.

\end{poem}

\begin{poem}{2012: Marťanská odysea}{Hurá do přírody}

\settowidth{\versewidth}{rudej byl v plamenech prej i mistr Jan Hus,}

Rudá je planeta, kde musíme hnít.\\*
Rudá je krev, až budeme dřít.\\*
Rudá je obloha, když odejde mrak,\\*
rudá je dobrá, tvrdil to Marx.

Rudý je uvnitř pečená kráva.\\*
Rudá je prej i ve vesmíru tráva.\\*
Rudá je louže, když šlápneš na sýčka,\\*
potom je rudý i naše moře a říčka.

Jako nálada, když zahrajou poslední kus,\\*
rudej byl v plamenech prej i mistr Jan Hus,\\*
je to barva, kterou, mám prostě rád,\\*
rudá je dobrá, vždyť je to Mars.



\end{poem}

\begin{poem}{Hvězdná brána: Stanoviště Alfa}{Hurá do přírody}

\settowidth{\versewidth}{že přišel za mnou Jack O'Neill,}

Já ač mám spánek bezesný,\\*
mně včera sen se zdál,\\*
že přišel za mnou Jack O'Neill,\\*
do SG-C mě vzal. 

Řekl: "Víte, Země stůně,\\*
Goa'uldi nám hrozbou jsou,\\*
vzepřít jim se neumíme,\\*
když bránou na nás jdou."

\end{poem}

\begin{poem}{Když se zamiluje kůň}{Zdeněk Svěrák, Jaroslav Uhlíř}

\settowidth{\versewidth}{láskou hlubokou jak tůň}

Když se zamiluje kůň\\*
tam někde v pastvinách,\\*
láskou hlubokou jak tůň\\*
tam někde v pastvinách.\\*
Když se zamiluje kůň\\*
koňskou láskou,\\*
zpívejte písničku\\*
pro jeho klisničku,\\*
nechte ho jít.

Když se zamiluje kůň\\*
tam někde v pastvinách,\\*
láskou hlubokou jak tůň\\*
tam někde v pastvinách.\\*
Když se zamiluje kůň\\*
koňskou láskou,\\*
zpívejte písničku\\*
pro jeho klisničku,\\*
nechte ho jít.

Nejkrásnější zvíře,\\*
zvíře pro rytíře\\*
jmenuje se kůň,\\*
jmenuje se kůň,\\*
Važte si ho, lidé,\\*
ať nám jich pár zbyde,\\*
jmenuje se kůň,\\*
jmenuje se kůň,\\*
jmenuje se kůň.

Slečna s bílou lysinkou,\\*
tam někde v pastvinách,\\*
bude brzy maminkou,\\*
tam někde v pastvinách.\\*
Vždyť se zamiloval kůň\\*
koňskou láskou,\\*
hřívu si navlnil, aby ji oslnil\\*
a cválá k ní.

Nejkrásnější zvíře,\\*
zvíře pro rytíře\\*
jmenuje se kůň,\\*
jmenuje se kůň.\\*
Važte si ho, lidé,\\*
ať nám jich pár zbyde,\\*
jmenuje se kůň,\\*
jmenuje se kůň,\\*
jmenuje se kůň.

Když se zamiluje kůň\\*
tam někde v pastvinách,\\*
láskou hlubokou jak tůň\\*
tam někde v pastvinách.\\*
Když se zamiluje kůň\\*
koňskou láskou,\\*
zpívejte písničku\\*
pro jeho klisničku,\\*
nechte ho jít.\\*
Zpívejte písničku\\*
pro jeho klisničku,\\*
nechte ho jít.

\end{poem}

\begin{poem}{Amerika}{Lucie}

\settowidth{\versewidth}{v duši zbylo světlo z jedný holky,}

Nandej mi do hlavy tvý brouky,\\*
a Bůh nám seber beznaděj,\\*
v duši zbylo světlo z jedný holky,\\*
tak mi teď za to vynadej.

Zima a promarněný touhy,\\*
do vrásek stromů padá déšť,\\*
Zbejvaj roky, asi ne moc dlouhý,\\*
do vlasů mi zabroukej:

pá pa pá pá\\*
pá pá pá pá\\*
pá pa pá pá oujeé\\*
pá pá pá pá\\*
pá pa pá pá oujeé

Tvoje voči jenom žhavý tóny,\\*
dotek slunce zapadá,\\*
horkej vítr rozezní mý zvony,\\*
do vlasů ti zabroukám:

pá pa pá pá\\*
pá pá pá pá\\*
pá pa pá pá oujeé\\*
pá pá pá pá\\*
pá pa pá pá oujeé

Na obloze křídla tažnejch ptáků,\\*
tak už na svý bráchy zavolej.\\*
Na tváře ti padaj slzy z mraků\\*
a Bůh nám sebral beznaděj. 

V duši zbylo světlo z jedný holky,\\*
do vrásek stromů padá déšť,\\*
poslední dny, hodiny a roky,\\*
do vlasů ti zabroukám:

pá pa pá pá\\*
pá pá pá pá\\*
pá pa pá pá oujeé\\*
pá pá pá pá\\*
pá pa pá pá oujeé

\end{poem}

\begin{poem}{Banánová šťáva}{Timudej}

\settowidth{\versewidth}{A tak si jede ten děda po mostě s valachem a sudy má za sebou.}

Tak jedu si po pauze, za sebou mám ty svý sudy,\\*
někde to vyložím, povezu zas prázdný sudy.\\*
Pojedu doleva, bohužel však nevím kudy\\*
a nebo doprava, doufám, že to bude tudy.

Olé olé olé olé a někdo tady je a kdo to asi je?\\*
Olé olé olé policie!\\*
Olé olé olé olé a někdo tady je a kdo to asi je?\\*
Olé olé olé policie!

A tak si jede ten děda po mostě s valachem a sudy má za sebou.\\*
Až dojede na asfaltovou silnici, kde ho staví četník...

Stát! Dopravní kontrola!\\*
Slezte z toho kozlíku a copak máte v těch sudech vzadu?!\\*
Tam mám banánovou šťávu, šťávu z banánů!\\*
Jo tak banánovou šťávu?!\\*
Tak to jsem teda ale ještě nikdy neslyšel! Takovou kravinu!\\*
To chci ochutnat tu vaši banánovou šťávu!\\*
To není problém, tady máte kalíšek a natočte si.\\*
Fuj, to je hnusný! Jak to někdo může pít?!\\*
To já nevím, to je šťáva z banánů! Banánová šťáva!!\\*
Běžte zpátky na ten kozlík a jeďte už ať vás tady nevidím!\\*
A budu počítat do čtyř! Jedna, dva, tři, čtyři!!\\*
Hyjé Banáne!!!

Olé olé olé olé a někdo tady je a kdo to asi je?\\*
Olé olé olé policie!\\*
Olé olé olé olé a někdo tady je a kdo to asi je?\\*
Olé olé olé policie!

\end{poem}

\begin{poem}{Malování}{Divokej Bill}

\settowidth{\versewidth}{jen táta a máma jsou s náma,}

Nesnaž se,\\*
znáš se.\\*
Řekni mi, co je jiný,\\*
jak v kleci máš se\\*
pro nevinný,\\*
noci dlouhý\\*
jsou plný touhy\\*
a lásky nás dvou.

Všechno hezký za sebou mám,\\*
můžu si za to sám.\\*
V hlavě hlavolam,\\*
jen táta a máma\\*
jsou s náma,\\*
napořád s náma.

To je to tvoje malování\\*
vzdušnejch zámků,\\*
malování\\*
po zdech holejma rukama\\*
tě nezachrání,\\*
už máš na kahánku,\\*
tě nezachrání,\\*
už seš na zádech.

Je to za náma,\\*
ty čteš poslední stránku,\\*
za náma,\\*
na zádech,\\*
za náma,\\*
už máš na kahánku,\\*
mezi náma,\\*
mi taky došel dech.

To je to tvoje malování\\*
vzdušnejch zámků,\\*
malování\\*
po zdech holejma rukama,\\*
tě nezachrání,\\*
už máš na kahánku,\\*
tě nezachrání,\\*
už seš na zádech.

Je to za náma,\\*
ty čteš poslední stránku,\\*
za náma,\\*
na zádech,\\*
za náma,\\*
už máš na kahánku,\\*
mezi náma,\\*
mi taky došel dech.

Znáš se.\\*
Řekni mi, co je jiný,\\*
jak v kleci máš se\\*
pro nevinný,\\*
noci dlouhý\\*
jsou plný touhy\\*
a lásky nás tří.

\end{poem}

\begin{poem}{Dej mi víc své lásky}{Olympic}

\settowidth{\versewidth}{Vymyslel jsem spoustu nápadu, aúú,}

Vymyslel jsem spoustu nápadu, aúú,\\*
co podporujou hloupou náladu, aúú,\\*
hodit klíče do kanálu,\\*
sjet po zadku holou skálu,\\*
v noci chodit strašit do hradu.

Dám si dvoje housle pod bradu, aúú,\\*
v bíle plachtě chodím pozadu, aúú,\\*
úplně melancholicky,\\*
s citem pro věc jako vždycky\\*
vyrábím tu hradní záhadu, aúú.

Má drahá, dej mi víc,\\*
má drahá, dej mi víc,\\*
má drahá, dej mi víc své lásky, aúú\\*
Ja nechci skoro nic,\\*
já nechci skoro nic,\\*
já chci jen pohladit tvé vlásky, aúú

Nejlepší z těch divnej nápadu, aúú,\\*
mi dokonale zvednul náladu, aúú,\\*
natrhám ti sedmikrásky,\\*
tebe celou s tvými vlásky\\*
zamknu si na sedm západů, aú.

Má drahá, dej mi víc,\\*
má drahá, dej mi víc,\\*
má drahá, dej mi víc své lásky, aúú\\*
Ja nechci skoro nic,\\*
já nechci skoro nic,\\*
já chci jen pohladit tvé vlásky, aúú

\end{poem}

\begin{poem}{Little Big Horn}{Greenhorns}

\settowidth{\versewidth}{Tam, kde leží Little Big Horn, je indiánská zem,}

Tam, kde leží Little Big Horn, je indiánská zem,\\*
tam přijíždí generál Custer se svým praporem,\\*
modrý kabáty jezdců, stíny dlouhejch karabin,\\*
a z indiánskejch signálů po nebi letí dým.

Říkal to Jim Bridger já měl jsem v noci sen,\\*
pod sedmou kavalerií jak krví rudne zem,\\*
kmen Siouxů je statečný a dobře svůj kraj zná,\\*
proč Custer neposlouchá ta slova varovná.

Tam blízko Little Big Hornu šedivou prérií\\*
táhne generál Custer se svou kavalerií,\\*
marně mu stopař Bridger radí: Zpátky povel dej!,\\*
Jedinou možnost ještě máš, život si zachovej!

Říkal to Jim Bridger já měl jsem v noci sen,\\*
pod sedmou kavalerií jak krví rudne zem,\\*
kmen Siouxů je statečný a dobře svůj kraj zná,\\*
proč Custer neposlouchá ta slova varovná.

Tam blízko Little Big Hornu se vznáší smrti stín,\\*
padají jezdci z koní, výstřely z karabin,\\*
límce modrejch kabátů barví krev červená,\\*
kmen Siouxů je statečný a dobře svůj kraj zná.

Říkal to Jim Bridger já měl jsem v noci sen,\\*
pod sedmou kavalerií jak krví rudne zem,\\*
kmen Siouxů je statečný a dobře svůj kraj zná,\\*
proč Custer neposlouchá ta slova varovná.

Pak všechno ztichlo a jen tamtam duní nad krajem,\\*
v oblacích prachu mizí Siouxů vítězný kmen,\\*
cáry vlajky hvězdnatý po kopcích vítr vál,\\*
tam uprostřed svých vojáků leží i generál.

Říkal to Jim Bridger já měl jsem v noci sen,\\*
pod sedmou kavalerií jak krví rudne zem,\\*
kmen Siouxů je statečný a dobře svůj kraj zná,\\*
proč Custer neposlouchal ta slova varovná.

\end{poem}

\begin{poem}{Vlak v 05}{Greenhorns}

\settowidth{\versewidth}{v němž jede parta, se kterou jsem chodil pít:}

Já ti řeknu, proč jsem dneska divnej tak,\\*
kdy' v dálce za tunelem začne houkat vlak,\\*
proč nechávám svou ciaretu vyhasnout,\\*
proč každou chvíli stojím jako solnej sloup.

Touhle tratí já jsem jezdil řadu let.\\*
Mašina a koleje byly můj svět\\*
a parta z vlaku, se kterou jsem chodil pít:\\*
strojvůdce Mike, výhybkář Joe a brzdař Kid.

Jednou, když jsme jeli z Friska v 05,\\*
tahle trať nebejvala žádnej med,\\*
po banánový slupce sklouz a z vlaku slít\\*
a pod kolama zůstal mladej Kid.

Za rok nato z Friska 05\\*
vyjel si Mike mašinfíra naposled.\\*
Tam v zatáčce za tunelem při srážce\\*
z budky vylít a zlomil si vaz o pražce.

Pak jsem už zbyl jen já a Výhybka Joe,\\*
ale neštěstí chodilo kolem nás dvou.\\*
Jednou Výhybka Joe nevrátil se zpět,\\*
ani on ani ten vlak v 05.

Takže už víš, proč jsem teda bledej tak,\\*
já slyším za tunelem houkat divnej vlak,\\*
v němž jede parta, se kterou jsem chodil pít:\\*
strojvůdce Mike, Výhybka Joe a brzdař Kid.

Ten vlak se ke mně řítí tmou,\\*
tak ahoj Kide, Miku, ahoj Joe...

\end{poem}

\begin{poem}{Blues hvízdavého vlaku}{Greenhorns}

\settowidth{\versewidth}{za pět minut má vyjet můj vlak,}

Poslouchej, jak ve stráni\\*
ozvěna šum kotle odráží,\\*
Poslouchej, jak ve stráni\\*
ozvěna šum kotle odráží,\\*
až uslyšíš hvízdání,\\*
to můj vlak vjede do nádraží.

Jen mou rakev klidně\\*
víkem zandavej,\\*
Jen mou rakev klidně\\*
víkem zandavej,\\*
kdo však dál teď bude\\*
řídit můj vlak hvízdavej?

Už je dvanáct nula pět,\\*
za pět minut má vyjet můj vlak,\\*
Už je dvanáct nula pět,\\*
za pět minut má vyjet můj vlak,\\*
kdo ho spustí ale vpřed,\\*
když já musím tu ležet naznak?

Jen mou rakev klidně\\*
na vůz nandavej,\\*
Jen mou rakev klidně\\*
na vůz nandavej,\\*
kdo však dál teď bude\\*
řídit můj vlak hvízdavej?

Dejte pryč ty věnce,\\*
strhejte ten černej flór,\\*
Dejte pryč ty věnce,\\*
strhejte ten černej flór,\\*
musím vyjet přece,\\*
zelenej mám semafor.

\end{poem}

\begin{poem}{Tři kříže}{Hop trop}

\settowidth{\versewidth}{ale odpouštět božský, snad mi tedy Bůh odpustí."}

Dávám sbohem břehům proklatejm,\\*
který v drápech má ďábel sám.\\*
Bílou přídí šalupa My Grave\\*
míří k útesům, který znám.

Jen tři kříže z bílýho kamení\\*
někdo do písku poskládal.\\*
Slzy v očích měl a v ruce znavený\\*
lodní deník, co sám do něj psal.

První kříž má pod sebou jen hřích,\\*
samý pití a rvačky jen,\\*
chřestot nožů, při kterým přejde smích,\\*
srdce kámen a jméno Sten.

Jen tři kříže z bílýho kamení\\*
někdo do písku poskládal.\\*
Slzy v očích měl a v ruce znavený\\*
lodní deník, co sám do něj psal.

Já Bob Green mám tváře zjizvený,\\*
štěkot psa zněl, když jsem se smál.\\*
Druhej kříž mám a spím pod zemí,\\*
že jsem falešný karty hrál.

Jen tři kříže z bílýho kamení\\*
někdo do písku poskládal.\\*
Slzy v očích měl a v ruce znavený\\*
lodní deník, co sám do něj psal.

Třetí kříž snad vyvolá jen vztek,\\*
Davy Rodgers těm dvoum život vzal\\*
Svědomí měl a vedle nich si klek:\\*
"Vím, trestat je lidský,\\*
ale odpouštět božský, snad mi tedy Bůh odpustí."

Jen tři kříže z bílýho kamení\\*
jsem jim do písku poskládal.\\*
Slzy v očích měl a v ruce znavený\\*
lodní deník a v něm, co jsem psal.

\end{poem}

\begin{poem}{Píseň zhrzeného trampa}{Jaromír Nohavica}

\settowidth{\versewidth}{Že prý se můžu vrátit zpět až dám se do cajku}

\begin{altverse}
Poněvadž nemám kanady
a neznám písně z pamp\\*
(johoho a neznám písně z pamp)\\*
vyloučili mě z osady
že prý jsem houby tramp\\*
(johoho že prý jsem houby tramp)
\end{altverse}

\begin{altverse}
Napsali si do cancáků
jen ať to každý ví\\*
(johoho jen ať to každý ví)\\*
Vyloučený z řad čundráků
ten frajer libový\\*
(johoho ten frajer libový)
\end{altverse}

Já jsem ostuda traperů\\*
já mám rád operu\\*
já mám rád jazz-rock\\*
chodím po světě bez nože\\*
to prý se nemože\\*
to prý jsem cvok\\*
já jsem nikdy neplul na šífu\\*
a všem šerifům jsem říkal Ba ne\\*
pane\\*
já jsem ostuda trempů\\*
já když chlempu\\*
tak v autokempu

\begin{altverse}
Povídal mi frajer Joe
jen žádný legrácky\\*
(johoho jen žádný legrácky)\\*
jinak chytneš na bendžo
čestný čundrácký\\*
(johoho a čestný čundrácký)
\end{altverse}

\begin{altverse}
Že prý se můžu vrátit zpět
až dám se do cajku\\*
(johoho až se dám do cajku)\\*
a odříkám jim nazpaměť
akordy na Vlajku\\*
(johoho akordy na Vlajku)
\end{altverse}

Já jsem ostuda traperů\\*
já mám rád operu\\*
já mám rád folk-rock\\*
chodím po světě bez nože\\*
to prý se nemože\\*
to prý jsem cvok\\*
já jsem nikdy neplul na šífu\\*
a všem šerifům jsem říkal Ba ne\\*
pane\\*
já jsem ostuda trempů\\*
já když chlempu\\*
tak v autokempu

\begin{altverse}
A tak chodím po světě
a mám zaracha\\*
(johoho a mám zaracha)\\*
na vandr chodím k Markétě
a dávám si bacha\\*
(johoho a dávam si bacha)
\end{altverse}

\begin{altverse}
Dokud se trampské úřady
nepoučí z chyb\\*
(johoho a nepoučí z chyb)\\*
zpívám si to svý nevadí
a zase bude líp\\*
(johoho a zase bude líp)
\end{altverse}

Já jsem ostuda traperů\\*
já mám rád operu\\*
já mám rád rock-rock\\*
chodím po světě bez nože\\*
to prý se nemože\\*
to prý jsem cvok\\*
já jsem nikdy neplul na šífu\\*
a všem šerifům jsem říkal Ba ne\\*
pane\\*
já jsem ostuda trempů\\*
já když chlempu\\*
tak v autokempu\\*
v tempu

\end{poem}

\begin{poem}{Grónská písnička}{Jaromír Nohavica}

\settowidth{\versewidth}{neboť medvěd jim předvádí spoustu fíglů.}

Daleko na severu je Grónská zem,\\*
žije tam Eskymačka s Eskymákem.\\*
My bychom umrzli, jim není zima,\\*
snídají nanuky a eskyma.\\*
My bychom umrzli, jim není zima,\\*
snídají nanuky a eskyma.

Mají se bezvadně, vyspí se moc,\\*
půl roku trvá tam polární noc.\\*
Na jaře vzbudí se a vyběhnou ven,\\*
půl roku trvá tam polární den.\\*
Na jaře vzbudí se a vyběhnou ven,\\*
půl roku trvá tam polární den.

Když sněhu napadne nad kotníky,\\*
hrávají s medvědy na četníky.\\*
Medvědi těžko jsou k poražení,\\*
neboť medvědy ve sněhu vidět není.\\*
Medvědi těžko jsou k poražení,\\*
neboť medvědy ve sněhu vidět není.

Pokaždé ve středu, přesně ve dvě\\*
zaklepe na na íglů hlavní medvěd.\\*
"Dobrý den, mohu dál na vteřinu?\\*
Nesu vám trochu ryb na svačinu."\\*
"Dobrý den, mohu dál na vteřinu?\\*
Nesu vám trochu ryb na svačinu."

V kotlíku bublá čaj, kamna hřejí,\\*
psi venku hlídají před zloději.\\*
Smíchem se otřásá celé iglů,\\*
neboť medvěd jim předvádí spoustu fíglů.\\*
Smíchem se otřásá celé iglů,\\*
neboť medvěd jim předvádí spoustu fíglů.

Tak žijou vesele na severu,\\*
srandu si dělají z teploměrů.\\*
My bychom umrzli, jim není zima,\\*
neboť jsou doma a mezi svýma.\\*
My bychom umrzli, jim není zima,\\*
neboť jsou doma a mezi svýma.
\end{poem}

\begin{poem}{Hlídač krav}{Jaromír Nohavica}

\settowidth{\versewidth}{a když je mi velmi smutno, lehnu do mokré trávy.}

Když jsem byl malý, říkali mi naši:\\*
Dobře se uč a jez chytrou kaši,\\*
až jednou vyrosteš, budeš doktorem práv.

Takový doktor si sedí pěkně v suchu,\\*
bere velký peníze a škrábe se v uchu.\\*
Já jim ale na to řek: Chci být hlídačem krav.

Já chci mít čapku s bambulí nahoře,\\*
jíst kaštany a mýt se v lavoře,\\*
od rána po celý den, zpívat si jen.\\*
Zpívat si:\\*
Pam pam pa dam pam padáda dam\\*
pam pam padam pam padádam\\*
pam padadadam padadadádam

K Vánocům mi kupovali hromady knih,\\*
co jsem ale vědět chtěl, to nevyčet jsem z nich,\\*
nikde jsem se nedozvěděl, jak se hlídají krávy.

Ptal jsem se starších a ptal jsem se všech,\\*
každý na mě hleděl jako na pytel blech,\\*
každý se mě opatrně tázal na moje zdraví.

Já chci mít čapku s bambulí nahoře,\\*
jíst kaštany a mýt se v lavoře,\\*
od rána po celý den, zpívat si jen.\\*
Zpívat si:\\*
Pam pam pa dam pam padáda dam\\*
pam pam padam pam padádam\\*
pam padadadam padadadádam

Teď už jsem starší a vím co vím,\\*
mnohé věci nemůžu a mnohé smím\\*
a když je mi velmi smutno, lehnu do mokré trávy.

S nohama křížem a rukama za hlavou,\\*
koukám nahoru na oblohu modravou,\\*
kde se mezi mraky honí moje strakaté krávy.

Já chci mít čapku s bambulí nahoře,\\*
jíst kaštany a mýt se v lavoře,\\*
od rána po celý den, zpívat si jen.\\*
Zpívat si:\\*
Pam pam pa dam pam padáda dam\\*
pam pam padam pam padádam\\*
pam padadadam padadadádam\\*
Zpívat si:\\*
Pam pam pa dam pam padáda dam\\*
pam pam padam pam padádam\\*
pam padadadam padadadádam

\end{poem}

\begin{poem}{Kometa}{Jaromír Nohavica}

\settowidth{\versewidth}{a o všech lidech, co kdy žili na téhle planetě.}

Spatřil jsem kometu, oblohou letěla,\\*
chtěl jsem jí zazpívat, ona mi zmizela,\\*
zmizela jako laň u lesa v remízku,\\*
v očích mi zbylo jen pár žlutých penízků.

Penízky ukryl jsem do hlíny pod dubem,\\*
až příště přiletí, my už tu nebudem,\\*
my už tu nebudem, ach, pýcho marnivá,\\*
spatřil jsem kometu, chtěl jsem jí zazpívat.

O vodě, o trávě, o lese,\\*
o smrti, se kterou smířit nejde se,\\*
o lásce, o zradě, o světě\\*
a o všech lidech, co kdy žili na téhle planetě.

Na hvězdném nádraží cinkají vagóny,\\*
pan Kepler rozepsal nebeské zákony,\\*
hledal, až nalezl v hvězdářských triedrech\\*
tajemství, která teď neseme na bedrech.

Velká a odvěká tajemství přírody,\\*
že jenom z člověka člověk se narodí,\\*
že kořen s větvemi ve strom se spojuje\\*
a krev našich nadějí vesmírem putuje.

Na na na\\*
Na ná na na na na\\*
Na na na\\*
Na ná na na na na

Spatřil jsem kometu, byla jak reliéf\\*
zpod rukou umělce, který už nežije,\\*
šplhal jsem do nebe, chtěl jsem ji osahat,\\*
marnost mne vysvlékla celého donaha.

Jak socha Davida z bílého mramoru\\*
stál jsem a hleděl jsem, hleděl jsem nahoru,\\*
až příště přiletí, ach, pýcho marnivá,\\*
my už tu nebudem, ale jiný jí zazpívá.

O vodě, o trávě, o lese,\\*
o smrti, se kterou smířit nejde se,\\*
o lásce, o zradě, o světě,\\*
bude to písnička o nás a kometě...

\end{poem}

\begin{poem}{Pochod marodů}{Jaromír Nohavica}

\settowidth{\versewidth}{ale už jsou nadranc, -dranc, -dranc,}

Krabička cigaret\\*
a do kafe rum, rum, rum,\\*
dvě vodky a Fernet\\*
a teď, doktore, čum, čum, čum,\\*
chrapot v hrudním koši,\\*
no to je zážitek,\\*
my jsme kámoši\\*
řidičů sanitek, -tek, -tek.

Měli jsme ledviny,\\*
ale už jsou nadranc, -dranc, -dranc,\\*
i tělní dutiny\\*
už ztratily glanc, glanc, glanc,\\*
u srdce divný zvuk,\\*
co je to, nemám šajn,\\*
a je to vlastně fuk,\\*
žijem fajn, žijem fajn, fajn, fajn.

Cirhóza, trombóza,\\*
dávivý kašel,\\*
tuberkulóza\\*
- jó, to je naše!\\*
neuróza, skleróza,\\*
ohnutá záda,\\*
paradentóza,\\*
no to je paráda!\\*
Jsme slabí na těle,\\*
ale silní na duchu,\\*
žijem vesele,\\*
juchuchuchuchu!

Už kolem nás chodí\\*
pepka mrtvice, -ce, -ce,\\*
tak pozor, marodi,\\*
je zlá velice, -ce, -ce,\\*
zná naše adresy\\*
a je to čiperka,\\*
koho chce, najde si,\\*
ten natáhne perka, -rka, -rka.

Zítra nás odvezou,\\*
bude veselo, -lo, -lo,\\*
medici vylezou\\*
na naše tělo, -lo, -lo,\\*
budou nám řezati\\*
ty naše vnitřnosti\\*
a přitom zpívati\\*
ze samé radosti, -sti, -sti.

Zpívati: cirhóza, trombóza,\\*
dávivý kašel,\\*
tuberkulóza,\\*
hele, já jsem to našel!\\*
Neuróza, skleróza,\\*
křivičná záda,\\*
paradentóza,\\*
no to je paráda!\\*
Byli slabí na těle,\\*
ale silní na duchu,\\*
žili vesele,\\*
než měli poru\\*
-chu -chu -chu -chu -chu -chu\\*
-chů -chu -chu -chu -chu\\*
-chu -chu -chu -chů -chu\\*
-chu -chu -chu -chu.

\end{poem}

\begin{poem}{Básnířka}{Jaromír Nohavica}

\settowidth{\versewidth}{Pak jednou v pondělí, přišla na koncert na koleje}

Mladičká básnířka s korálky nad kotníky \\*
Bouchala na dvířka paláce poetiky\\*
S někým se vyspala, někomu nedala \\*
Láska jako hobby \\*
Tak o tom napsala sonet na čtyři doby 

Svoje srdce skloňovala podle vzoru Ferlinghetti \\*
Ve vzduchu nechávala viset vždy jen půlku věty \\*
Plná tragiky, plná mystiky \\*
Plná splínu \\*
Tak jí to otiskli v jednom magazínu 

Bývala viděna v malém baru u Rozhlasu \\*
od sebe kolena a cizí ruka kolem pasu \\*
Trochu se napila, trochu se opila \\*
Na účet redaktora \\*
Za týden na to byla hvězdou mikrofóra 

Pod paží nosila rozepsané rukopisy \\*
Ráno se budila vedle záchodové mísy. \\*
Múzou políbena, životem potřísněná\\* 
Plná zázraků \\*
A pak ji vyhodili z gymplu a hned nato i z baráku 

Šly řeči okolím, že měla něco se esenbáky \\*
Ať bylo cokoliv, přestala věřit na zázraky \\*
Cítila u srdce, jak po ní přešla \\*
Železná bota \\*
Tak o tom napsala sonet, a ten byl ze života 

Pak jednou v pondělí, přišla na koncert na koleje \\*
A hlasem pokorným prosila o text Darmoděje \\*
Péro vzala a pak se dala \\*
Tichounce do pláče \\*
/:A její slzy kapaly na její mrkváče:/ 

\end{poem}

\begin{poem}{Já viděl divoké koně}{Jaromír Nohavica}

\settowidth{\versewidth}{Běželi běželi bez uzdy a sedla}

Já viděl divoké koně, \\*
běželi soumrakem.\\*
Já viděl divoké koně,\\*
běželi soumrakem.\\*
Vzduch těžký byl a divně voněl\\*
tabákem.\\*
Vzduch těžký byl a divně voněl\\*
tabákem.

Běželi běželi bez uzdy a sedla\\*
krajinou řek a hor.\\*
Běželi běželi bez uzdy a sedla\\*
krajinou řek a hor.\\*
Sper to čert jaká touha je to vedla\\*
za obzor.\\*
Sper to čert jaká touha je to vedla\\*
za obzor.

Snad vesmír nad vesmírem, \\*
snad lístek na věčnost.\\*
Snad vesmír nad vesmírem, \\*
snad lístek na věčnost.\\*
Naše touho ještě neumírej,\\* 
sil máme dost.\\*
Naše touho ještě neumírej, \\*
sil máme dost.

V nozdrách sládne zápach klisen \\*
na břehu jezera.\\*
V nozdrách sládne zápach klisen \\*
na břehu jezera.\\*
Milování je divoká píseň \\*
večera.\\*
Milování je divoká píseň \\*
večera.

Stébla trávy sklání hlavu, \\*
staví se do šiku.\\*
Stébla trávy sklání hlavu, \\*
staví se do šiku.\\*
Král s dvořany přijíždí na popravu \\*
zbojníků.\\*
Král s dvořany přijíždí na popravu \\*
zbojníků.

Chtěl bych jak divoký kůň běžet běžet,\\* 
nemyslet na návrat.\\*
Chtěl bych jak divoký kůň běžet běžet, \\*
nemyslet na návrat.\\*
S koňskými handlíři vyrazit dveře, \\*
to bych rád.\\*
S koňskými handlíři vyrazit dveře, \\*
to bych rád.

\end{poem}

\begin{poem}{Tři čuníci}{Jaromír Nohavica}

\settowidth{\versewidth}{vyšli prostě do světa a vesele si zpívají:}

V řadě za sebou tři čuníci jdou,\\*
ťápají si v blátě cestou-necestou,\\*
kufry nemají, cestu neznají,\\*
vyšli prostě do světa a vesele si zpívají: 

Ui ui ui ui uí\\*
ui ui ui ui uí\\*
ui ui ui ui uí uí\\*
ui ui ui ui uí

Ui ui ui ui uí\\*
ui ui ui ui uí\\*
ui ui ui ui ui ui ui ui\\*
ui ui ui ui ui ui uí

Auta jezdí tam, náklaďáky sem,\\*
tři čuníci jdou, jdou rovnou za nosem,\\*
ušima bimbají, žito křoupají,\\*
vyšli prostě do světa a vesele si zpívají: 

Ui ui ui ui uí\\*
ui ui ui ui uí\\*
ui ui ui ui uí uí\\*
ui ui ui ui uí

Ui ui ui ui uí\\*
ui ui ui ui uí\\*
ui ui ui ui ui ui ui ui\\*
ui ui ui ui ui ui uí

Levá, pravá - teď!, přední, zadní - už!,\\*
tři čuníci jdou, jdou jako jeden muž,\\*
lidé zírají, důvod neznají,\\*
proč ti malí čuníci tak vesele si zpívají: 

Ui ui ui ui uí\\*
ui ui ui ui uí\\*
ui ui ui ui uí uí\\*
ui ui ui ui uí

Ui ui ui ui uí\\*
ui ui ui ui uí\\*
ui ui ui ui ui ui ui ui\\*
ui ui ui ui ui ui uí

Když kopýtka pálí, když jim dojde dech,\\*
sednou ke studánce na vysoký břeh,\\*
ušima bimbají, kopýtka máchají,\\*
chvilinku si odpočinou, a pak dál se vydají: 

Ui ui ui ui uí\\*
ui ui ui ui uí\\*
ui ui ui ui uí uí\\*
ui ui ui ui uí

Ui ui ui ui uí\\*
ui ui ui ui uí\\*
ui ui ui ui ui ui ui ui\\*
ui ui ui ui ui ui uí

Když se spustí déšť, roztrhne se mrak,\\*
k sobě přitisknou se čumák na čumák,\\*
blesky bleskají, kapky pleskají,\\*
oni v dešti, nepohodě vesele si zpívají: 

Ui ui ui ui uí\\*
ui ui ui ui uí\\*
ui ui ui ui uí uí\\*
ui ui ui ui uí

Ui ui ui ui uí\\*
ui ui ui ui uí\\*
ui ui ui ui ui ui ui ui\\*
ui ui ui ui ui ui uí

Za tu spoustu let, co je světem svět,\\*
přešli zeměkouli třikrát tam a zpět\\*
v řadě za sebou, hele, támhle jdou,\\*
pojďme s nima zazpívat si jejich píseň veselou: 

Ui ui ui ui uí\\*
ui ui ui ui uí\\*
ui ui ui ui uí uí\\*
ui ui ui ui uí

Ui ui ui ui uí\\*
ui ui ui ui uí\\*
ui ui ui ui ui ui ui ui\\*
ui ui ui ui ui ui uí

\end{poem}

\begin{poem}{Udavač z Těšína}{Jaroslav Hutka}

\settowidth{\versewidth}{vtíravým hlasem svým bude nám zpívat}

Udavač z Těšína, koncert mu začíná\\*
vtíravým hlasem svým bude nám zpívat\\*
v sále se zhasíná, dám si sklenku vína\\*
s publikem budu se na něj též dívat

Jak je ten udavač krásný\\*
jak velký je umělec\\*
výraz má zřetelně jasný\\*
dnes nám zpívá zbabělec

Zasněné obrazy, láska z nich vychází\\*
básnické obraty, půvabné rýmy\\*
Dívky jsou raněné, ženy jsou zmámené\\*
a muži ztrácejí pod sebou zem

Jak je ten udavač krásný\\*
jak velký je umělec\\*
výraz má zřetelně jasný\\*
dnes nám zpívá zbabělec

Kritik si notuje, na písních hoduje\\*
sál zpívá chytlavý, radostný refrén\\*
Poslouchám pozorně, už je mi odporně\\*
odněkud slyším zpěv homérských Sirén

Jak je ten udavač krásný\\*
jak velký je umělec\\*
výraz má zřetelně jasný\\*
dnes nám zpívá zbabělec

Na cestě do Vídně zachoval se bídně\\*
písničkáře Kryla jidášsky objal\\*
Klíče mu věnoval, pak ho fízlům udal\\*
zatlouká zatlouká a tím mě dojal

Jak je ten udavač krásný\\*
jak velký je umělec\\*
výraz má zřetelně jasný\\*
dnes nám zpívá zbabělec

Tam za bolševiků zpíval všem morálku\\*
hrdina před všemi prvního řádu\\*
Kritikou oceněn, anděly odměněn\\*
bohatě co blíží se ke svému pádu

\end{poem}

\begin{poem}{Maruška}{Malomocnost prázdnoty}

\settowidth{\versewidth}{Co jste to za vojsko,}

\begin{altverse}
Na malém plácku\\*
Na malém plácku\\*
děti si hrají,\\*
děti si hrají,\\*
hrají si na válku,\\*
hrají si na válku,\\*
všechno už mají.\\*
všechno už mají.
\end{altverse}

\begin{altverse}
Stejnokroj z tepláků\\*
Stejnokroj z tepláků\\*
větší než na míru,\\*
větší než na míru,\\*
dřevěný pistole,\\*
dřevěný pistole,\\*
čepice z papíru.\\*
čepice z papíru.
\end{altverse}

\begin{altverse}
A bitva za bitvou,\\*
A bitva za bitvou,\\*
tak to jde dokola,\\*
tak to jde dokola,\\*
až potom najednou\\*
až potom najednou\\*
něčí hlas zavolá:\\*
něčí hlas zavolá:
\end{altverse}

\vfill\eject

\begin{altverse}
Zastavte válku\\*
Zastavte válku\\*
Maruška brečí,\\*
Maruška brečí,\\*
dostala kamenem\\*
dostala kamenem\\*
při naší zteči.\\*
při naší zteči.
\end{altverse}

\begin{altverse}
Co jste to za vojsko,\\*
Co jste to za vojsko,\\*
když místo střílení\\*
když místo střílení\\*
do svých nepřátel\\*
do svých nepřátel\\*
házíte kamení.\\*
házíte kamení.
\end{altverse}

\begin{altverse}
Na to my nehrajem,\\*
Na to my nehrajem,\\*
vy nám to kazíte,\\*
vy nám to kazíte,\\*
nedbáte pravidel,\\*
nedbáte pravidel,\\*
/:a pak se divíte!\\*
A pak se divíte!/:
\end{altverse}

\begin{altverse}
Na velkém plácku\\*
Na velkém plácku\\*
hrají si dospělí,\\*
hrají si dospělí,\\*
jen místo dřevěných\\*
jen místo dřevěných\\*
hračky maj z oceli.\\*
hračky maj z oceli.
\end{altverse}

\begin{altverse}
Ocel se zarývá\\*
Ocel se zarývá\\*
do kůry stromů,\\*
do kůry stromů,\\*
desítky jizviček\\*
desítky jizviček\\*
a blesky hromů.\\*
a blesky hromů.
\end{altverse}

\begin{altverse}
Některým na duši,\\*
Některým na duši,\\*
některým do těla\\*
některým do těla\\*
vpálí znamení,\\*
vpálí znamení,\\*
proč nikdo nevolá?\\*
proč nikdo nevolá?
\end{altverse}

\begin{altverse}
Zastavte válku\\*
Zastavte válku\\*
Maruška brečí,\\*
Maruška brečí,\\*
dostala kamenem\\*
dostala kamenem\\*
při naší zteči.\\*
při naší zteči.
\end{altverse}

\begin{altverse}
Co jste to za vojsko,\\*
Co jste to za vojsko,\\*
když místo střílení\\*
když místo střílení\\*
do svých nepřátel\\*
do svých nepřátel\\*
házíte kamení.\\*
házíte kamení.
\end{altverse}

\begin{altverse}
Na to my nehrajem,\\*
Na to my nehrajem,\\*
vy nám to kazíte,\\*
vy nám to kazíte,\\*
nedbáte pravidel,\\*
nedbáte pravidel,\\*
/:a pak se divíte!\\*
A pak se divíte!/:
\end{altverse}

\end{poem}

\begin{poem}{Špinavej Titanik}{Malomocnost prázdnoty}

\settowidth{\versewidth}{snad ke štěstí, snad do záhuby}

Všude kolem prázdno,\\*
loď pluje v úplném bezvětří,\\*
v podpalubí bahno,\\*
na palubě špína.\\*
Čert ví, kam pluje,\\*
snad ke štěstí, snad do záhuby,\\*
na kupách hnoje\\*
smrtka se na ně zubí.

Opilý námořník\\*
s opilou děvkou\\*
na lodi souloží\\*
pod mokrou plachtou.\\*
Nic kolem nevidí,\\*
nic nevnímají,\\*
jenom se houpají\\*
na vlnách chtíče.

Na týhletý lodi\\*
s kormidlem v půli zlomeným\\*
cestující chodí\\*
tváří se štastně, jsou veselí!\\*
Všichni jsou slepí,\\*
svůj osud radši netuší,\\*
nevidí útesy\\*
nasáklý smrtí.

\vfill\eject

Opilý námořník\\*
s opilou děvkou\\*
na lodi souloží\\*
pod mokrou plachtou.\\*
Nic kolem nevidí,\\*
nic nevnímají,\\*
jenom se houpají\\*
na vlnách chtíče.

Trosky lodi plavou\\*
po mořské hladině v bezvětří,\\*
moře plné lidí,\\*
na břeh nikoho už nepustí!\\*
Obrovské bohatství,\\*
sobecky nastřádané,\\*
nikdo už nevlastní,\\*
válí se všude po dně,\\*
jen...

Opilý námořník\\*
s opilou děvkou\\*
na lodi souloží\\*
pod mokrou plachtou.\\*
Nic kolem nevidí,\\*
nic nevnímají,\\*
jenom se houpají\\*
na vlnách chtíče.

\end{poem}

\begin{poem}{Měsíc}{Mňága a Žďorp}

\settowidth{\versewidth}{teďka svítí měsíc pro každýho zvlášť mně je to líto}

Děkuju ti za to, žes mi ještě zavolala\\*
moje duše černá už to ani nečekala.\\*
Jenže moje drahá, je to všechno trochu na nic\\*
já už jsem se rozhod - zítra budu znovu panic. 

Děkuju ti za to, žes to rovnou nepoložil,\\*
že jsi nezapomněl, co jsi se mnou všechno prožil.\\*
Teďka svítí měsíc pro každýho zvlášť mně je to líto\\*
jenže už je konec však víš to...\\*
...vím to!

měsíc\\*
svítí měsíc

měsíc\\*
svítí měsíc

měsíc\\*
svítí měsíc

měsíc\\*
svítí měsíc

\end{poem}

\begin{poem}{Mezi horami}{Čechomor}

\settowidth{\versewidth}{hned na hrob padla a viac něvstala}

Mezi horami\\*
lipka zelená,\\*
mezi horami\\*
lipka zelená,\\*
zabili Janka, Janíčka, Janka\\*
miesto jeleňa,\\*
zabili Janka, Janíčka, Janka\\*
miesto jeleňa.

Keď ho zabili,\\*
zamordovali,\\*
keď ho zabili,\\*
zamordovali,\\*
na jeho hrobě, na jeho hrobě\\*
kříž postavili,\\*
na jeho hrobě, na jeho hrobě\\*
kříž postavili.

Ej křížu, křížu\\*
ukřižovaný,\\*
ej křížu, křížu\\*
ukřižovaný,\\*
zde leží Janík, Janíček, Janík,\\*
zamordovaný,\\*
zde leží Janík, Janíček, Janík,\\*
zamordovaný.

\vfill\eject

Tu šla Anička\\*
plakat Janíčka,\\*
tu šla Anička\\*
plakat Janíčka,\\*
hned na hrob padla a viac něvstala\\*
dobrá Anička,\\*
hned na hrob padla a viac něvstala\\*
dobrá Anička.

Mezi horami... 
\end{poem}

\begin{poem}{Jižní kříž}{Jan Nedvěd}

\settowidth{\versewidth}{světem protloukal ses, jak ten život pádí,}

Spí Jižní kříž,\\*
jak říkali jsme hvězdám kdysi v mládí,\\*
to na studený zemi\\*
ještě uměli jsme milovat a spát.

A dál, však to znáš,\\*
světem protloukal ses, jak ten život pádí,\\*
dneska písničky třeba vod Červánků\\*
dojmou tě, jak vrátil bys' to rád.

Zase toulal by ses Foglarovým rájem\\*
a stavěl Bobří hráz,\\*
se smečkou vlků čekal na jaro,\\*
jak stejská se, až po zádech jde mráz.

Spí Jižní kříž,\\*
vidíš všechna místa, kde jsi někdy byl,\\*
to když, naplněnej smutkem,\\*
jsi plakal, plakal nebo snil.

Zase toulal by ses Foglarovým rájem\\*
a stavěl Bobří hráz,\\*
se smečkou vlků čekal na jaro,\\*
jak stejská se, až po zádech jde mráz.
\end{poem}

\begin{poem}{Krysař}{Znouzectnost}

\settowidth{\versewidth}{a hodně dlouho v tomhle městě byl cítit vzduch křivdou.}

Bylo nebylo, kde se to tu vzalo,\\*
černé na bílém to na plakátech stálo.\\*
Bude tu hrát krysař a ty jeho krysy\\*
a bude to muzika dobrá jako kdysi.

Bylo nebylo, pak na scéně stáli,\\*
potkat je v noci ve městě, možná byste se báli.\\*
Středověký škorně a staletou halenu,\\*
do copánků spletený vlasy barvy havranů.

Bylo nebylo, bylo nebylo,\\*
bylo nebylo a bylo nebylo.\\*
A bylo. \\*
Bylo nebylo, bylo nebylo,\\*
bylo nebylo a bylo nebylo.

Bylo nebylo, kde se vlastně vzali\\*
a tou kouzelnou muzikou štěstí rozdávali.\\*
Pak ten okamžik skončil a nikdo nechtěl domů,\\*
atmosféra, kdo tam nebyl, nevěřil by tomu.

Bylo nebylo, bylo nebylo,\\*
bylo nebylo a bylo nebylo.\\*
A bylo. \\*
Bylo nebylo, bylo nebylo,\\*
bylo nebylo a bylo nebylo.

Bylo nebylo, uplynul ňákej čas\\*
a plakáty na nárožích visely tu zas.\\*
Lidi zachvátila horečka a každý tam chtěl jít,\\*
bez rozdílu názorů zas to kouzlo prožít.

Bylo nebylo, bylo nebylo,\\*
bylo nebylo a bylo nebylo.\\*
A bylo. \\*
Bylo nebylo, bylo nebylo,\\*
bylo nebylo a bylo nebylo.

Bylo nebylo, prošly divný zprávy,\\*
že vyšlehnou hranice pro čerty a ďábly.\\*
Do datumu na plakátech chybělo pár dní\\*
a na radnici Krysaře zvou si páni radní.

Bylo nebylo, bylo nebylo,\\*
bylo nebylo a bylo nebylo.\\*
Nebylo. \\*
Bylo nebylo (nebylo), bylo nebylo,\\*
bylo nebylo a bylo nebylo.

Bylo nebylo, řekli, to by teda nešlo,\\*
s touhle vaší vizáží hrát budete tu těžko!\\*
Nás to vůbec nezajímá, že vás mají lidi rádi,\\*
my jsme tady od toho jejich mysl chránit!

Bylo nebylo, bylo nebylo,\\*
bylo nebylo a bylo nebylo.\\*
Nebylo. \\*
Bylo nebylo (nebylo), bylo nebylo,\\*
bylo nebylo a bylo nebylo.

Bylo nebylo, jako v tý pohádce,\\*
Krysař město opouští a hudba zní v dálce.\\*
Jak za kouzelnou píšťalou za ním myšlenky táhnou\\*
a hodně dlouho v tomhle městě byl cítit vzduch křivdou.

Bylo nebylo a bylo nebylo,\\*
bylo nebylo a bylo nebylo.\\*
Nebylo. \\*
Bylo nebylo (nebylo), bylo nebylo,\\*
bylo nebylo a bylo nebylo.
\end{poem}

\begin{poem}{Anděl}{Karel Kryl}

\settowidth{\versewidth}{Když novinky mi sděloval}

Z rozmlácenýho kostela\\*
v krabici s kusem mýdla,\\*
přinesl jsem si anděla,\\*
polámali mu křídla.

Díval se na mě oddaně,\\*
já měl jsem trochu trému.\\*
Tak vtiskl jsem mu do dlaně\\*
lahvičku od parfému.

A proto, prosím, věř mi,\\*
chtěl jsem ho žádat,\\*
aby mi mezi dveřmi\\*
pomohl hádat,\\*
co mě čeká\\*
a nemine,\\*
co mě čeká\\*
a nemine.

Pak hlídali jsme oblohu,\\*
pozorujíce ptáky,\\*
debatujíce o Bohu\\*
a hraní na vojáky.

Do tváře jsem mu neviděl,\\*
pokoušel se ji schovat,\\*
to asi ptákům záviděl,\\*
že mohou poletovat.

A proto, prosím, věř mi,\\*
chtěl jsem ho žádat,\\*
aby mi mezi dveřmi\\*
pomohl hádat,\\*
co mě čeká\\*
a nemine,\\*
co mě čeká\\*
a nemine.

Když novinky mi sděloval\\*
u okna do ložnice,\\*
já křídla jsem mu ukoval\\*
z mosazný nábojnice.

A tak jsem pozbyl anděla,\\*
on oknem uletěl mi,\\*
však přítel prý mi udělá\\*
novýho z mojí helmy.

A proto, prosím, věř mi,\\*
chtěl jsem ho žádat,\\*
aby mi mezi dveřmi\\*
pomohl hádat,\\*
co mě čeká\\*
a nemine,\\*
co mě čeká\\*
a nemine.

\end{poem}

\begin{poem}{Hvězdář}{UDG}

\settowidth{\versewidth}{v bělostných šatech pro nemocné,}

Ztrácíš se před očima\\*
rosteš jen ve vlastním stínu.\\*
Každá další vina\\*
odkrývá moji vinu.

Ztrácíš se před očima,\\*
rosteš jen ve vlastním stínu.\\*
Každá další vina\\*
odkrývá moji vinu.

\begin{altverse}
Ve vínu dávno nic nehledám, \\*
(Ve vínu dávno nic nehledám) \\*
nehledám.
\end{altverse}
\begin{altverse}
Ve vínu dávno nic nehledám, \\*
(Ve vínu dávno nic nehledám) \\*
nehledám.
\end{altverse}

Jak luna mizí s nocí\\*
v bělostných šatech pro nemocné,\\*
prosit je zvláštní pocit,\\*
jen, ať je den, noc ne.

Jak luna mizí s nocí\\*
v milostných šatech pro nemocné,\\*
prosit je zvláštní pocit,\\*
jen, ať je den, noc ne.

\begin{altverse}
Od proseb dávno nic nečekám, \\*
(Od proseb dávno nic nečekám) \\*
nečekám.
\end{altverse}
\begin{altverse}
Od proseb dávno nic nečekám, \\*
(Od proseb dávno nic nečekám) \\*
nečekám.
\end{altverse}

Na chodbách v bludných kruzích \\*
zářivka vyhasíná,\\*
a já ti do infuzí \\*
chci přilít trochu vína.

Na nebi jiných sluncí, \\*
jak se tam asi cítíš,\\*
s nebeskou interpunkcí, \\*
jiným tulákům svítíš.

\begin{altverse}
Ve vínu dávno nic nehledám, \\*
(Ve vínu dávno nic nehledám) \\*
nehledám.
\end{altverse}
\begin{altverse}
Ve vínu dávno nic nehledám, \\*
(Ve vínu dávno nic nehledám) \\*
nehledám.
\end{altverse}

Jak luna mizí s nocí\\*
v bělostných šatech pro nemocné,\\*
prosit je zvláštní pocit,\\*
jen, ať je den, noc ne.

Jak luna mizí s nocí\\*
v milostných šatech pro nemocné,\\*
prosit je zvláštní pocit,\\*
jen, ať je den, noc ne.

Obzor než klesne níž, \\*
je ráno a ty spíš.\\*
Od vlků odraná, \\*
hvězdáře Giordana.

Obzor než klesne níž, \\*
je ráno a ty spíš.\\*
Od vlků odraná, \\*
hvězdáře Giordana.

Obzor než klesne níž, \\*
je ráno a ty spíš.\\*
Od vlků odraná, \\*
hvězdáře Giordana.
\end{poem}

\begin{poem}{Bedna vod whisky}{Hoboes\\(a nějakej Slovák, co moc nerozumí rýmům)}

\settowidth{\versewidth}{a chlastal bych tam s Billem a chlastal by tam Ben.}

Dneska už mi fóry ňák nejdou přes pysky,\\*
stojím s dlouhou kravatou na bedně od whisky.\\*
Stojím s dlouhým obojkem, jak stájovej pinč,\\*
tu kravatu co nosím mi navlík soudce Lynč.

Tak kopni do tý bedny, ať panstvo nečeká,\\*
jsou dlouhý schody do nebe a štreka daleká.\\*
Do nebeskýho báru, já sucho v krku mám,\\*
tak kopni do tý bedny, ať na cestu se dám.

Mít tak všechny bedny od whisky vypitý,\\*
postavil bych malej dům na louce ukrytý.\\*
Postavil bych malej dům a z vokna koukal ven\\*
a chlastal bych tam s Billem a chlastal by tam Ben.

Tak kopni do tý bedny, ať panstvo nečeká,\\*
jsou dlouhý schody do nebe a štreka daleká.\\*
Do nebeskýho báru, já sucho v krku mám,\\*
tak kopni do tý bedny, ať na cestu se dám.

Když jsem štípnul koně a ujel jen pár mil,\\*
nechtěl běžet, dokavád se whisky nenapil,\\*
zatracená smůla zlá, zatracenej pech,\\*
když kůň cucá whisku jak u potoka mech.

Tak kopni do tý bedny, ať panstvo nečeká,\\*
jsou dlouhý schody do nebe a štreka daleká.\\*
Do nebeskýho báru, já sucho v krku mám,\\*
tak kopni do tý bedny, ať na cestu se dám.

Kdyby si se hochu jen pořád nechtěl rvát,\\*
nemusel jsi dneska na týhle bedně stát.\\*
Moh si někde v suchu tu svojí whisky pít,\\*
nemusel jsi dneska na krku laso mít.

Tak kopni do tý bedny, ať panstvo nečeká,\\*
jsou dlouhý schody do nebe a štreka daleká.\\*
Do nebeskýho báru, já sucho v krku mám,\\*
tak kopni do tý bedny, ať na cestu se dám.

Kdyby jsi se, chlapčik, len nechtěl dobirat,\\*
nemusel jsi těraz na hentenonej stat,\\*
moh' jsi někde v krčme tu palenicu piť,\\*
nemusel jsi chlapčik s oprazom co maš.

Tak kopni do tý bedny, ať panstvo nečeká,\\*
jsou dlouhý schody do nebe a štreka daleká.\\*
Do nebeskýho báru, já sucho v krku mám,\\*
tak kopni do tý bedny, ať na cestu se dám.

Až kopneš do tý bedny, jak se to dělává,\\*
do krku mi vostane jen dírka mrňavá.\\*
Jenom dírka mrňavá a k smrti jenom krok,\\*
mám to smutnej konec a whisky ani lok.

Tak kopni do tý bedny, ať panstvo nečeká,\\*
jsou dlouhý schody do nebe a štreka daleká.\\*
Do nebeskýho báru, já sucho v krku mám,\\*
tak kopni do tý bedny...

\end{poem}

% TOP %
%\renewcommand*{\topname}{Anglické} % Name for the table of songs
%\maketop

\section{Čobolácké}
Předmluva k slovenským songům. Teda k slovenskému songu, protože víc jich
nemají... 

\newpage
\thispagestyle{empty}

\begin{poem}{L. A. G. Song}{Horkýže Slíže}

\settowidth{\versewidth}{tak otvor branu lebo ju rozbijem!}

Poď sem, nech sa s tebou zblížim.\\*
Poď sem, ja ti neubližim.\\*
Poď sem, ja ťa nezbijem.\\*
I'm sorry, I'm a Lesbian.

Poď sem, som tvoj dvorny basnik.\\*
Mam energiu za dvanastich\\*
a k tomu lubrikačný gel.\\*
I'm sorry, I'm a really Gay.

Lesbian's and Gay's song.\\*
Lesbian's and Gay's sooooong.\\*
Lesbian's and Gay's song.\\*
Lesbian's and Gay's song.

Poď sem, pustim Iron Maiden.\\*
Poď sem, nalejem ti za jeden.\\*
Poď sem, s barskym nepijem.\\*
I'm sorry, I'm a Lesbian.

Poď sem, tu si ku mne hačaj.\\*
Keď som ťa pozval na rum a čaj.\\*
Stoj! Nechod nikam! Čo je? Hej!\\*
I'm sorry, I'm a really Gay.

Lesbian's and Gay's song.\\*
Lesbian's and Gay's sooooong.\\*
Lesbian's and Gay's song.\\*
Lesbian's and Gay's song.

Poď sem, spolu mame v plane.\\*
že ťa okupem vo fontane.\\*
tak otvor branu lebo ju rozbijem!\\*
I'm sorry baby, I'm a Lesbian.

Viem, neopakuj mi to stale\\*
ja som vyrastol na Death Metale,\\*
ja takymto veciam rozumiem.\\*
I'm sorry... ale veď ja viem!

Lesbian's and Gay song.\\*
Lesbian's and Gay sooooong.\\*
Lesbian's and Gay song.\\*
Lesbian's and Gay song.

\end{poem}

\section{Anglické}
Předmluva k anglickým songům. 

\newpage
\thispagestyle{empty}

% \mainmatter

\begin{poem}{Ring of Fire}{Johnny Cash}

\settowidth{\versewidth}{I fell into a burnin' ring of fire}

\incipit{Love is a burnin' thing}\\*
And it makes a firery ring\\*
Bound by wild desire\\*
I fell into a ring of fire

I fell into a burnin' ring of fire\\*
I went down, down, down\\*
And the flames went higher\\*
And it burns, burns, burns\\*
The ring of fire, the ring of fire

\end{poem}

\begin{poem}{Over the Rainbow}{Judy Garland}

\settowidth{\versewidth}{Wake up where the clouds are far behind me}

Somewhere over the rainbow\\*
Way up high\\*
And the dreams that you dream of\\*
Once in a lullaby

Somewhere over the rainbow\\*
Bluebirds fly\\*
And the dreams that you dream of\\*
Dreams really do come true

Someday, I wish upon a star\\*
Wake up where the clouds are far behind me\\*
Where trouble melts like lemon drops\\*
High above the chimney top\\*
That's where you'll find me

Somewhere over the rainbow\\*
Bluebirds fly\\*
And the dreams that you dare to\\*
Oh why, oh why can't I?

\end{poem}

\begin{poem}{Toxicity}{System of a Down}

\settowidth{\versewidth}{Flashlight reveries caught in the headlights of a truck}

Conversion, software version 7.0\\*
Looking at life through the eyes of a tired hub\\*
Eating seeds as a pastime activity\\*
The toxicity of our city, our city

You, what do you own the world?\\*
How do you own disorder, disorder\\*
Now somewhere between the sacred silence\\*
Sacred silence and sleep\\*
Somewhere, between the sacred silence and sleep\\*
Disorder, disorder, disorder

More wood for the fires, loud neighbors\\*
Flashlight reveries caught in the headlights of a truck\\*
Eating seeds as a pastime activity\\*
The toxicity of our city, of our city

Now, what do you own the world?\\*
How do you own disorder, disorder\\*
Now somewhere between the sacred silence\\*
Sacred silence and sleep\\*
Somewhere between the sacred silence and sleep\\*
Disorder, disorder, disorder

You, what do you own the world?\\*
How do you own disorder, disorder\\*
Now somewhere between the sacred silence\\*
Sacred silence and sleep\\*
Somewhere, between the sacred silence and sleep\\*
Disorder, disorder, disorder

When I became the sun\\*
I shone life into the man's hearts\\*
When I became the sun\\*
I shone life into the man's hearts

\end{poem}

\begin{poem}{Jack of Hands}{Edward Ka-Spel}

\settowidth{\versewidth}{He’s free to tread and lift the sheets}

Charity begins at homes\\*
For those too sick, those indisposed \\*
Footsteps in the corridor \\*
Little Johnny’s playing dead

They gave the Jack of Hands the key \\*
He’s free to tread and lift the sheets \\*
He’ll be dressed in white they said

But I don’t believe in angels

When the good son rises \\*
When we hear the birds \\*
Little Johnny's curled up and crying \\*
Alas a touch disturbed

The Jack of Hands is far away \\*
He’s opening a school \\*
They’ll offer him a chariot \\*
But Jack prefers to walk 

Dangerous

\vfill\eject

\begin{altverse}
Hush now \\*
Hush now \\*
Jack is coming \\*
Jack is coming \\*
Shush now \\*
Shush now \\*
Like you’re sleeping \\*
Like you’re sleeping\\*
\end{altverse}
He’s carrying his special box \\*
No one ever asks what’s in it \\*
No more suffer little children \\*
\begin{altverse}
Jack’ll fix it \\*
Jack’ll fix it 
\end{altverse}


It’s Jack the national treasure \\*
Jack’s beating on his chest \\*
He’s flying off to Africa \\*
Cause Africa’s the best 

They’ll cordon off the airport \\*
Because the crowd is vast \\*
And Jack will play the trinity \\*
He’ll be home at last 

Still I don’t believe in angels

\end{poem}

\begin{poem}{Drunken Sailor}{Lidová}

\settowidth{\versewidth}{Put him in the bed with the captains daughter,}

What shall we do with the drunken sailor?\\*
What shall we do with the drunken sailor?\\*
What shall we do with the drunken sailor?\\*
Early in the morning!

Way hey and up she rises,\\*
way hey and up she rises,\\*
way hey and up she rises,\\*
early in the morning!

Shave his belly with a rusty razor,\\*
shave his belly with a rusty razor,\\*
shave his belly with a rusty razor,\\*
early in the morning!

Way hey and up she rises,\\*
way hey and up she rises,\\*
way hey and up she rises,\\*
early in the morning!

Put him in a long boat till his sober,\\*
put him in a long boat till his sober,\\*
put him in a long boat till his sober,\\*
early in the morning!

Way hey and up she rises,\\*
way hey and up she rises,\\*
way hey and up she rises,\\*
early in the morning!

Stick him in a barrel with a hosepipe on him,\\*
stick him in a barrel with a hosepipe on him,\\*
stick him in a barrel with a hosepipe on him,\\*
early in the morning!

Way hey and up she rises,\\*
way hey and up she rises,\\*
way hey and up she rises,\\*
early in the morning!

Put him in the bed with the captain's daughter,\\*
put him in the bed with the captain's daughter,\\*
put him in the bed with the captain's daughter,\\*
early in the morning!

Way hey and up she rises,\\*
way hey and up she rises,\\*
way hey and up she rises,\\*
early in the morning!

That’s what we do with the drunken sailor,\\*
that’s what we do with the drunken sailor,\\*
that’s what we do with the drunken sailor,\\*
early in the morning!

Way hey and up she rises,\\*
way hey and up she rises,\\*
way hey and up she rises,\\*
early in the morning!

Way hey and up she rises,\\*
way hey and up she rises,\\*
way hey and up she rises,\\*
early in the morning!

\end{poem}

\begin{poem}{Old Maui}{Lidová}

\settowidth{\versewidth}{Cause we're homeward bound from the Arctic ground}

'Tis a damn tough life full of toil and strife\\*
We whalermen undergo\\*
And we don't give a damn when the day is done\\*
How hard the winds did blow\\*
Cause we're homeward bound from the Arctic ground\\*
With a good ship, taut and free\\*
And we don't give a damn when we drink our rum\\*
With the girls of Old Maui

Rolling down to Old Maui, me boys\\*
Rolling down to Old Maui\\*
We're homeward bound from the Arctic ground\\*
Rolling down to Old Maui

Once more we sail with a northerly gale\\*
Towards our island home\\*
Our mainmast sprung, our whaling done\\*
And we ain't got far to roam\\*
Six hellish months we passed away\\*
On the cold Kamchatka Sea\\*
But now we're bound from the Arctic ground\\*
Rolling down to Old Maui

Rolling down to Old Maui, me boys\\*
Rolling down to Old Maui\\*
We're homeward bound from the Arctic ground\\*
Rolling down to Old Maui

Once more we sail with a northerly gale\\*
Through the ice and wind and rain\\*
Them coconut fronds, them tropical lands\\*
We soon shall see again\\*
Even now their big brown eyes look out\\*
Hoping some fine day to see\\*
Our baggy sails running 'fore the gale\\*
Rolling down to Old Maui

Rolling down to Old Maui, me boys\\*
Rolling down to Old Maui\\*
We're homeward bound from the Arctic ground\\*
Rolling down to Old Maui

How soft the breeze through the island trees\\*
Now the ice is far astern\\*
Them native maids, them tropical glades\\*
Is a-waiting our return\\*
I will rant and roar and row to shore\\*
And paint them beaches red\\*
Waking in the arms of a Wahine maid\\*
With a big fat aching head

Rolling down to Old Maui, me boys\\*
Rolling down to Old Maui\\*
We're homeward bound from the Arctic ground\\*
Rolling down to Old Maui

Rolling down to Old Maui, me boys\\*
Rolling down to Old Maui\\*
We're homeward bound from the Arctic ground\\*
Rolling down to Old Maui

\end{poem}

\begin{poem}{Roll the Woodpile Down}{Lidová}

\settowidth{\versewidth}{Rollin', rollin', rollin' the whole world 'round}

Away down South where the cocks do crow\\*
Way down in Florida\\*
Them girls all dance to the old banjo\\*
And we'll roll the woodpile down

Rollin', rollin', rollin' the whole world 'round\\*
That brown gal of mine's on the Georgia line\\*
And we'll roll the woodpile down

Oh, what can you do in Tampa bay?\\*
Way down in Florida\\*
But give them yellow girls all your pay\\*
And we'll roll the woodpile down

Rollin', rollin', rollin' the whole world 'round\\*
That brown gal of mine's on the Georgia line\\*
And we'll roll the woodpile down

Them Cardiff girls ain't got no frills\\*
Way down in Florida\\*
They're skinny and tight as catfish gills\\*
And we'll roll the woodpile down

Rollin', rollin', rollin' the whole world 'round\\*
That brown gal of mine's on the Georgia line\\*
And we'll roll the woodpile down

Oh, why do them little girls love me so?\\*
Way down in Florida\\*
Because I don't tell all I know\\*
And we'll roll the woodpile down

Rollin', rollin', rollin' the whole world 'round\\*
That brown gal of mine's on the Georgia line\\*
And we'll roll the woodpile down

Oh, one more pull and that will do\\*
Way down in Florida\\*
For we're the boys to kick her through\\*
And we'll roll the woodpile down

Rollin', rollin', rollin' the whole world 'round\\*
That brown gal of mine's on the Georgia line\\*
And we'll roll the woodpile down\\*
That brown gal of mine's on the Georgia line\\*
And we'll roll the woodpile down

\end{poem}

\begin{poem}{Rocky Road to Dublin}{Lidová}

\settowidth{\versewidth}{And I have frightened all the dogs on the rocky road
to Dublin,}

While in the merry month of May from me home I started\\*
Left the girls of Tuam nearly broken-hearted\\*
Saluted father dear, kissed me darlin' Mother\\*
Drank a pint of beer me grief and tears to smother\\*
Then off to reap the corn, and leave where I was born\\*
Cut a stout blackthorn to banish ghost and goblin,\\*
In a bran' new pair of brogues I rattled o'er the bogs\\*
And frightened all the dogs on the rocky road to Dublin,\\*
One, two, three, four, five. 

Hunt the hare and turn her\\*
Down the rocky road, and all the ways to Dublin\\*
Whack fol-lol-de-ra.

In Mullingar that night I rested limbs so weary,\\*
Started by daylight next morning light and airy,\\*
Took a drop of the pure, to keep my heart from \mbox{sinking},\\*
That's the Paddy's cure, whene'er he's on drinking,\\*
To see the lasses smile, laughing all the while,\\*
At my curious style, 'twould set your heart a-bubblin'\\*
They ax'd if I was hired, the wages I required,\\*
Till I was almost tired of the rocky road to Dublin.\\*
One, two, three, four, five. 

Hunt the hare and turn her\\*
Down the rocky road, and all the ways to Dublin\\*
Whack fol-lol-de-ra.

In Dublin next arrived, I thought it such a pity,\\*
To be so soon deprived a view of that fine city,\\*
Then I took a stroll out among the quality,\\*
My bundle it was stole in a neat locality;\\*
Something crossed my mind, then I looked behind,\\*
No bundle could I find upon me stick a-wobblin',\\*
Enquiring for the rogue, they said my Connaught brogue\\*
Wasn't much in vogue on the rocky road to Dublin.\\*
One, two, three, four, five. 

Hunt the hare and turn her\\*
Down the rocky roaad, and all the ways to Dublin\\*
Whack fol-lol-de-ra.

From there I got away my spirits never failing,\\*
Landed on the quay as the ship was sailing,\\*
Captain at me roared, said that no room had he,\\*
When I jumped aboard, a cabin found for Paddy\\*
Down among the pigs, I played some funny rigs\\*
Danced some hearty jigs, the water round me \mbox{bubblin}\\*
When off to Holyhead I wished myself was dead,\\*
Or better far, instead, on the rocky road to Dublin.\\*
One, two, three, four, five. 

Hunt the hare and turn her\\*
Down the rocky roaad, and all the ways to Dublin\\*
Whack fol-lol-de-ra.

The boys of Liverpool, when we safely landed,\\*
Called myself a fool, I could no longer stand it;\\*
Blood began to boil, temper I was losin'\\*
Poor old Erin's isle they began abusin'\\*
"Hurrah my soul!" sez I, my shillelagh I let fly,\\*
Some Galway boys were by, saw I was a hobble in,\\*
Then with a loud Hurrah, they joined in the affray,\\*
We quickly cleared the way, for the rocky road to Dublin.\\*
One, two, three, four, five. 

Hunt the hare and turn her\\*
Down the rocky road, and all the ways to Dublin\\*
Whack fol-lol-de-ra.

\end{poem}

\end{document}\grid
