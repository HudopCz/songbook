\documentclass[10.5pt]{book}

\usepackage{verse}
\usepackage{tocloft}

\usepackage[utf8]{inputenc}
%\usepackage[czech]{babel}
\PassOptionsToPackage{bookmarks, colorlinks=false, hidelinks}{hyperref}
% Use PoetryTeX; http://www.ctan.org/pkg/poetrytex
\usepackage[numberpoems, clearpageafterpoem, useincipits]{poetrytex}

% Use the PA5 paper size
\usepackage[paperwidth=140mm,paperheight=210mm]{geometry}

\renewcommand{\pttitle}{Zpěvník}
\renewcommand{\ptsubtitle}{hudop.cz}
\renewcommand{\ptauthor}{}
\renewcommand{\ptdate}{2017/07/30}
\renewcommand{\ptdedication}{Made by HUDOP:\\*
This is dedicated to\\*
Someone else, not you.}
\renewcommand{\contentsname}{Osnova}

\newcommand\tab[1][1cm]{\hspace*{#1}}

\begin{document}

\maketitle
\makededication

% Number pages with small roman numerals (i, ii, iii, iv...)
% \frontmatter
% Start numbering pages with normal arabic numerals.
\mainmatter

% TOC %
\maketoc

% TOP %
\renewcommand*{\topname}{Songlist} % Name for the table of songs
\maketop

\section{České}

Česko je vnitrozemský stát ležící ve střední Asii. Sousedí na západě
s Německem, na východě není daleko do Ruska. Na mapách nenaleznete žádné moře,
jen krev a mléko, uran, strdí a údajně i med. V 90. letech minulého století
začaly plány na zatopení části Moravy tak, aby se sousední Slovensko stalo
českou zámořskou kolonií. Kvůli Greenpeace k tomu však nedošlo (Moravistánští
katolíci to přisuzují vlivu sv. Cyrila a Metoděje, jistá část populace zase
Klementa Gottwalda).

Již starověcí Egypťané, Číňané a Olmékové zmiňovali vzdálený cizí kraj, kde
údajně poklidně žil kulturně vyspělý národ – národ Čechů. Čechové, původem
Slované, zakládali knihovny a čas trávili studiem, aby pak večer měli o čem
diskutovat v hospodě u piva. Bojovali neradi, raději trousili moudra při lidové
zábavě, případně něco vyráběli či kutili. Po první návštěvě kupce Sáma se věci
naučili i prodávat. Časem se naučili také prodej věcí, které se jim tak docela
nepovedou. 

Čeština je extrémně složitý jazyk, proto už Češi nemají kapacity na to, aby se
naučili nějaký cizí jazyk. Je obecně známou skutečností, že největší šance měl
při volbě nového Papeže kardinál Vlk, který ale nakonec neuspěl z jediného
důvodu a to, že nikdo nedokázal vyslovit jeho jméno. Ve skutečnosti je v
Čechistánu mnoho jazyků, ovšem jen čeština je úřední. 

Hymna Česka je stará píseň původem z předminulého století. Její melodie i text
pochází z veselohry, která se hrávala pro pobavení prostého lidu v divadlech na
pražských předměstích. Její text, melodii, videoklip a další informace najdete
na oficiálním kanálu YouTube. Autorem textu je Josef Šroub a hudbu složil Josef
Kajetán Pterodaktyl. 

\newpage
\thispagestyle{empty}

% Start numbering pages with normal arabic numerals.
% \mainmatter

\begin{poem}{Za pokladem Černé perly}{Hurá do přírody}

\settowidth{\versewidth}{"Hodokvas vítězný - hřeš, jak jen chceš."}

Na bárce z krunýře pějem svůj verš,\\*
jak se tak ploužíme líně jak veš.\\*
Velrybu bílou znal pražský král,\\*
pro něj ji chytit nám za úkol dal.\\*
"Hodokvas vítězný - hřeš, jak jen chceš."

Zemřela chýra - nás přepad' žal, \\*
tělo nechť plaví proud po vlnách v dál.\\*
Výprava její se skončila v půli,\\*
toť dílo krysí a krysám jen k vůli,\\*
přísahu porušil ten sprostý král.

Do moře hladem pad' nejeden pták,\\*
shůry až do vln, jak stáh' by ho hák.\\*
Keblovský hlídkař se svědectvím černým,\\*
jenž v plášti hlídkuje nad mořem věrným,\\*
došel až za známé hranice map.

Na bárce z krunýře přes moře jdem, \\*
vstříc krajům mlžným, kde skrytá je zem.\\*
Nespočet na moři spatříme rán,\\*
smát se i plakat zří nás oceán,\\*
dřív, nežli vstoupíme na rodnou zem,\\*
dřív, nežli vstoupíme na naši rodnou zem.

\end{poem}

\begin{poem}{2012: Marťanská odysea}{Hurá do přírody}

\settowidth{\versewidth}{rudej byl v plamenech prej i mistr Jan Hus,}

Rudá je planeta, kde musíme hnít.\\*
Rudá je krev, až budeme dřít.\\*
Rudá je obloha, když odejde mrak,\\*
rudá je dobrá, tvrdil to Marx.

Rudý je uvnitř pečená kráva.\\*
Rudá je prej i ve vesmíru tráva.\\*
Rudá je louže, když šlápneš na sýčka,\\*
potom je rudý i naše moře a říčka.

Jako nálada, když zahrajou poslední kus,\\*
rudej byl v plamenech prej i mistr Jan Hus,\\*
je to barva, kterou, mám prostě rád,\\*
rudá je dobrá, vždyť je to Mars.

\end{poem}

\begin{poem}{Hvězdná brána: Stanoviště Alfa}{Hurá do přírody}

\settowidth{\versewidth}{že přišel za mnou Jack O'Neill,}

Já ač mám spánek bezesný,\\*
mně včera sen se zdál,\\*
že přišel za mnou Jack O'Neill,\\*
do SG-C mě vzal. 

Řekl: "Víte, Země stůně,\\*
Goa'uldi nám hrozbou jsou,\\*
vzepřít jim se neumíme,\\*
když bránou na nás jdou."

\end{poem}

\begin{poem}{Amerika}{Lucie}

\settowidth{\versewidth}{v duši zbylo světlo z jedný holky,}

Nandej mi do hlavy tvý brouky,\\*
a Bůh nám seber beznaděj,\\*
v duši zbylo světlo z jedný holky,\\*
tak mi teď za to vynadej.

Zima a promarněný touhy,\\*
do vrásek stromů padá déšť,\\*
Zbejvaj roky, asi ne moc dlouhý,\\*
do vlasů mi zabroukej:

pá pa pá pá\\*
pá pá pá pá\\*
pá pa pá pá oujeé\\*
pá pá pá pá\\*
pá pa pá pá oujeé

Tvoje voči jenom žhavý tóny,\\*
dotek slunce zapadá,\\*
horkej vítr rozezní mý zvony,\\*
do vlasů ti zabroukám:

pá pa pá pá\\*
pá pá pá pá\\*
pá pa pá pá oujeé\\*
pá pá pá pá\\*
pá pa pá pá oujeé

Na obloze křídla tažnejch ptáků,\\*
tak už na svý bráchy zavolej.\\*
Na tváře ti padaj slzy z mraků\\*
a Bůh nám sebral beznaděj. 

V duši zbylo světlo z jedný holky,\\*
do vrásek stromů padá déšť,\\*
poslední dny, hodiny a roky,\\*
do vlasů ti zabroukám:

pá pa pá pá\\*
pá pá pá pá\\*
pá pa pá pá oujeé\\*
pá pá pá pá\\*
pá pa pá pá oujeé

\end{poem}

\begin{poem}{Anděl}{Karel Kryl}

\settowidth{\versewidth}{Když novinky mi sděloval}

Z rozmlácenýho kostela\\*
v krabici s kusem mýdla,\\*
přinesl jsem si anděla,\\*
polámali mu křídla.

Díval se na mě oddaně,\\*
já měl jsem trochu trému.\\*
Tak vtiskl jsem mu do dlaně\\*
lahvičku od parfému.

A proto, prosím, věř mi,\\*
chtěl jsem ho žádat,\\*
aby mi mezi dveřmi\\*
pomohl hádat,\\*
co mě čeká\\*
a nemine,\\*
co mě čeká\\*
a nemine.

Pak hlídali jsme oblohu,\\*
pozorujíce ptáky,\\*
debatujíce o Bohu\\*
a hraní na vojáky.

Do tváře jsem mu neviděl,\\*
pokoušel se ji schovat,\\*
to asi ptákům záviděl,\\*
že mohou poletovat.

A proto, prosím, věř mi,\\*
chtěl jsem ho žádat,\\*
aby mi mezi dveřmi\\*
pomohl hádat,\\*
co mě čeká\\*
a nemine,\\*
co mě čeká\\*
a nemine.

Když novinky mi sděloval\\*
u okna do ložnice,\\*
já křídla jsem mu ukoval\\*
z mosazný nábojnice.

A tak jsem pozbyl anděla,\\*
on oknem uletěl mi,\\*
však přítel prý mi udělá\\*
novýho z mojí helmy.

A proto, prosím, věř mi,\\*
chtěl jsem ho žádat,\\*
aby mi mezi dveřmi\\*
pomohl hádat,\\*
co mě čeká\\*
a nemine,\\*
co mě čeká\\*
a nemine.

\end{poem}

\begin{poem}{Banánová šťáva}{Timudej}

\settowidth{\versewidth}{A tak si jede ten děda po mostě s valachem a sudy má za sebou.}

Tak jedu si po pauze, za sebou mám ty svý sudy,\\*
někde to vyložím, povezu zas prázdný sudy.\\*
Pojedu doleva, bohužel však nevím kudy\\*
a nebo doprava, doufám, že to bude tudy.

Olé olé olé olé a někdo tady je a kdo to asi je?\\*
Olé olé olé policie!\\*
Olé olé olé olé a někdo tady je a kdo to asi je?\\*
Olé olé olé policie!

A tak si jede ten děda po mostě s valachem a sudy má za sebou.\\*
Až dojede na asfaltovou silnici, kde ho staví četník...

Stát! Dopravní kontrola!\\*
Slezte z toho kozlíku a copak máte v těch sudech vzadu?!\\*
Tam mám banánovou šťávu, šťávu z banánů!\\*
Jo tak banánovou šťávu?!\\*
Tak to jsem teda ale ještě nikdy neslyšel! Takovou kravinu!\\*
To chci ochutnat tu vaši banánovou šťávu!\\*
To není problém, tady máte kalíšek a natočte si.\\*
Fuj, to je hnusný! Jak to někdo může pít?!\\*
To já nevím, to je šťáva z banánů! Banánová šťáva!!\\*
Běžte zpátky na ten kozlík a jeďte už ať vás tady nevidím!\\*
A budu počítat do čtyř! Jedna, dva, tři, čtyři!!\\*
Hyjé Banáne!!!

Olé olé olé olé a někdo tady je a kdo to asi je?\\*
Olé olé olé policie!\\*
Olé olé olé olé a někdo tady je a kdo to asi je?\\*
Olé olé olé policie!

\end{poem}

\begin{poem}{Batalion}{Spirituál kvintet}

\settowidth{\versewidth}{Víno na kuráž a k ránu dvě hodinky spánku,}

Víno máš a markytánku,\\*
dlouhá noc se prohýří.\\*
Víno máš a chvilku spánku,\\*
díky, díky verbíři.

Dříve než se rozední,\\*
kapitán k osedlání rozkaz dává,\\*
ostruhami do slabin koně pohání.\\*
Tam na straně polední,\\*
čekají ženy, zlaťáky a sláva,\\*
do výstřelů z karabin zvon už vyzvání..

Ref.:\\*
Víno na kuráž a pomilovat markytánku,\\*
zítra do Burgund Batalion zamíří.\\*
Víno na kuráž a k ránu dvě hodinky spánku,\\*
díky, díky Vám královští verbíři.

Rozprášen je Batalion,\\*
poslední vojáci se k zemi hroutí,\\*
na polštáři z kopretin budou věčně spát.\\*
Neplač sladká Marion,\\*
verbíři nové chlapce přivedou ti,\\*
za královský hermelín padne každý rád.

Ref.

Víno máš a markytánku,\\*
dlouhá noc se prohýří.\\*
Víno máš a chvilku spánku,\\*
díky, díky verbíři...

\end{poem}

\begin{poem}{Básnířka}{Jaromír Nohavica}

\settowidth{\versewidth}{Pak jednou v pondělí, přišla na koncert na koleje}

Mladičká básnířka s korálky nad kotníky \\*
Bouchala na dvířka paláce poetiky\\*
S někým se vyspala, někomu nedala \\*
Láska jako hobby \\*
Tak o tom napsala sonet na čtyři doby 

Svoje srdce skloňovala podle vzoru Ferlinghetti \\*
Ve vzduchu nechávala viset vždy jen půlku věty \\*
Plná tragiky, plná mystiky \\*
Plná splínu \\*
Tak jí to otiskli v jednom magazínu 

Bývala viděna v malém baru u Rozhlasu \\*
od sebe kolena a cizí ruka kolem pasu \\*
Trochu se napila, trochu se opila \\*
Na účet redaktora \\*
Za týden na to byla hvězdou mikrofóra 

Pod paží nosila rozepsané rukopisy \\*
Ráno se budila vedle záchodové mísy. \\*
Múzou políbena, životem potřísněná\\* 
Plná zázraků \\*
A pak ji vyhodili z gymplu a hned nato i z baráku 

Šly řeči okolím, že měla něco se esenbáky \\*
Ať bylo cokoliv, přestala věřit na zázraky \\*
Cítila u srdce, jak po ní přešla \\*
Železná bota \\*
Tak o tom napsala sonet, a ten byl ze života 

Pak jednou v pondělí, přišla na koncert na koleje \\*
A hlasem pokorným prosila o text Darmoděje \\*
Péro vzala a pak se dala \\*
Tichounce do pláče \\*
/:A její slzy kapaly na její mrkváče:/ 

\end{poem}

\begin{poem}{Bedna vod whisky}{Hoboes\\(a nějakej Slovák, co moc nerozumí rýmům)}

\settowidth{\versewidth}{a chlastal bych tam s Billem a chlastal by tam Ben.}

Dneska už mi fóry ňák nejdou přes pysky,\\*
stojím s dlouhou kravatou na bedně od whisky.\\*
Stojím s dlouhým obojkem, jak stájovej pinč,\\*
tu kravatu co nosím mi navlík soudce Lynč.

Tak kopni do tý bedny, ať panstvo nečeká,\\*
jsou dlouhý schody do nebe a štreka daleká.\\*
Do nebeskýho báru, já sucho v krku mám,\\*
tak kopni do tý bedny, ať na cestu se dám.

Mít tak všechny bedny od whisky vypitý,\\*
postavil bych malej dům na louce ukrytý.\\*
Postavil bych malej dům a z vokna koukal ven\\*
a chlastal bych tam s Billem a chlastal by tam Ben.

Tak kopni do tý bedny, ať panstvo nečeká,\\*
jsou dlouhý schody do nebe a štreka daleká.\\*
Do nebeskýho báru, já sucho v krku mám,\\*
tak kopni do tý bedny, ať na cestu se dám.

Když jsem štípnul koně a ujel jen pár mil,\\*
nechtěl běžet, dokavád se whisky nenapil,\\*
zatracená smůla zlá, zatracenej pech,\\*
když kůň cucá whisku jak u potoka mech.

Tak kopni do tý bedny, ať panstvo nečeká,\\*
jsou dlouhý schody do nebe a štreka daleká.\\*
Do nebeskýho báru, já sucho v krku mám,\\*
tak kopni do tý bedny, ať na cestu se dám.

Kdyby si se hochu jen pořád nechtěl rvát,\\*
nemusel jsi dneska na týhle bedně stát.\\*
Moh si někde v suchu tu svojí whisky pít,\\*
nemusel jsi dneska na krku laso mít.

Tak kopni do tý bedny, ať panstvo nečeká,\\*
jsou dlouhý schody do nebe a štreka daleká.\\*
Do nebeskýho báru, já sucho v krku mám,\\*
tak kopni do tý bedny, ať na cestu se dám.

Kdyby jsi se, chlapčik, len nechtěl dobirat,\\*
nemusel jsi těraz na hentenonej stat,\\*
moh' jsi někde v krčme tu palenicu piť,\\*
nemusel jsi chlapčik s oprazom co maš.

Tak kopni do tý bedny, ať panstvo nečeká,\\*
jsou dlouhý schody do nebe a štreka daleká.\\*
Do nebeskýho báru, já sucho v krku mám,\\*
tak kopni do tý bedny, ať na cestu se dám.

Až kopneš do tý bedny, jak se to dělává,\\*
do krku mi vostane jen dírka mrňavá.\\*
Jenom dírka mrňavá a k smrti jenom krok,\\*
mám to smutnej konec a whisky ani lok.

Tak kopni do tý bedny, ať panstvo nečeká,\\*
jsou dlouhý schody do nebe a štreka daleká.\\*
Do nebeskýho báru, já sucho v krku mám,\\*
tak kopni do tý bedny...

\end{poem}

\begin{poem}{Bláznova ukolébavka}{Jan Nedvěd}

\settowidth{\versewidth}{Máš má ovečko dávno spát, dnes máme půlnoc fakt temnou.}

Máš má ovečko dávno spát, i píseň ptáků končí.\\*
Kvůli nám přestal vítr vát, jen můra zírá zvenčí.\\*
Já znám její zášť, tak vyhledej skrýš,\\*
zas má bílej plášť a v okně je mříž.

Máš má ovečko dávno spát\\*
a můžeš hřát, ty mě můžeš hřát.\\*
Vždyť přijdou se ptát, zítra zas přijdou se ptát,\\*
jestli ty v mých představách už mizíš.

Máš má ovečko dávno spát, dnes máme půlnoc\\*
\tab temnou.\\*
Ráno budou nám bláznů lát, že ráda snídáš se mnou.\\*
Proč měl bych jim lhát, že jsem tady sám,\\*
když Tebe mám rád, když Tebe tu mám.

Máš má ovečko dávno spát\\*
a můžeš hřát, ty mě můžeš hřát.\\*
Vždyť přijdou se ptát, zítra zas přijdou se ptát,\\*
jestli ty v mých představách už mizíš.

Máš má ovečko dávno spát\\*
a můžeš hřát, ty mě můžeš hřát.\\*
Vždyť přijdou se ptát, zítra zas přijdou se ptát,\\*
jestli ty v mých představách už mizíš...

\end{poem}

\begin{poem}{Blues Folsomské věznice}{Greenhorns}

\settowidth{\versewidth}{pravdu měla máma, radila:"Nechoď s tou holkou!",}

Můj děda bejval blázen, texaskej ahasver,\\*
a na půdě nám po něm zůstal ošoupanej kvér,\\*
ten kvér obdivovali všichni kámoši z okolí\\*
a máma mi říkala: "Nehraj si s tou pistolí!"

Jenže i já byl blázen, tak zralej pro malér,\\*
a ze zdi jsem sundával tenhleten dědečkův kvér,\\*
pak s kapsou vyboulenou chtěl jsem bejt chlap all right\\*
a s holkou vykutálenou hrál jsem si na Bonnie and Clyde.

Ale udělat banku, to není žádnej žert,\\*
sotva jsem do ní vlítnul, hned zas vylít' jsem jak čert,\\*
místo jako kočka já utíkám jak slon,\\*
takže za chvíli mě veze policejní anton.

Teď okno mřížovaný mi říká, že je šlus,\\*
proto tu ve věznici zpívám tohle Folsom Blues.\\*
pravdu měla máma, radila:"Nechoď s tou holkou!",\\*
a taky mi říkala:"Nehraj si s tou pistolkou!" 

\end{poem}

\begin{poem}{Blues hvízdavého vlaku}{Greenhorns}

\settowidth{\versewidth}{za pět minut má vyjet můj vlak,}

Poslouchej, jak ve stráni\\*
ozvěna šum kotle odráží,\\*
Poslouchej, jak ve stráni\\*
ozvěna šum kotle odráží,\\*
až uslyšíš hvízdání,\\*
to můj vlak vjede do nádraží.

Ref.:\\*
Jen mou rakev klidně\\*
víkem zandavej,\\*
Jen mou rakev klidně\\*
víkem zandavej,\\*
kdo však dál teď bude\\*
řídit můj vlak hvízdavej?

Už je dvanáct nula pět,\\*
za pět minut má vyjet můj vlak,\\*
Už je dvanáct nula pět,\\*
za pět minut má vyjet můj vlak,\\*
kdo ho spustí ale vpřed,\\*
když já musím tu ležet naznak?

Ref.

Dejte pryč ty věnce,\\*
strhejte ten černej flór,\\*
Dejte pryč ty věnce,\\*
strhejte ten černej flór,\\*
musím vyjet přece,\\*
zelenej mám semafor.

\end{poem}

\begin{poem}{Colorado}{Kabát}

\settowidth{\versewidth}{Já radši utratil jsem psa a všechny prachy}

Táta vždycky říkal: Hochu žádný strachy\\*
jseš cowboy, v Coloradu můžeš krávy pást\\*
Já radši utratil jsem psa a všechny prachy\\*
do srdce Evropy já vodjel v klidu krást.

Narvaný kapsy prsteny, řetězy zlatý\\*
tam kolem krku místní indiáni maj\\*
a ty co nemakaj, tak jsou nejvíc bohatý\\*
musim si pohnout, dokavaď tam rozdávaj.

Ref.:\\*
Z Billa na Nováka změnim si svý jméno\\*
a až tu malou zemi celou rozkradem\\*
\tab (rozkradem)\\*
tak se vrátím ve svý rodný Colorado\\*
a o tý zlatý žíle řeknu doma všem.

Tam kradou všichni, co blízko vokolo bydlej\\*
šerif se na ně jenom hezky usmívá\\*
kdyby se nesmál, tak ho okamžitě zmydlej\\*
házej mu kosti za to, že se nedívá.

Místo krav tam, nelžu vám, prej pasou holky\\*
a když jim nezaplatíš, vyrazej ti dech\\*
ale s IQ to tam nebude tak horký\\*
místo na koních tam jezděj v medvědech.

Ref.

Ref.

\end{poem}

\begin{poem}{Dej mi víc své lásky}{Olympic}

\settowidth{\versewidth}{Vymyslel jsem spoustu nápadu, aúú,}

Vymyslel jsem spoustu nápadu, aúú,\\*
co podporujou hloupou náladu, aúú,\\*
hodit klíče do kanálu,\\*
sjet po zadku holou skálu,\\*
v noci chodit strašit do hradu.

Dám si dvoje housle pod bradu, aúú,\\*
v bíle plachtě chodím pozadu, aúú,\\*
úplně melancholicky,\\*
s citem pro věc jako vždycky\\*
vyrábím tu hradní záhadu, aúú.

Má drahá, dej mi víc,\\*
má drahá, dej mi víc,\\*
má drahá, dej mi víc své lásky, aúú\\*
Ja nechci skoro nic,\\*
já nechci skoro nic,\\*
já chci jen pohladit tvé vlásky, aúú

Nejlepší z těch divnej nápadu, aúú,\\*
mi dokonale zvednul náladu, aúú,\\*
natrhám ti sedmikrásky,\\*
tebe celou s tvými vlásky\\*
zamknu si na sedm západů, aú.

Má drahá, dej mi víc,\\*
má drahá, dej mi víc,\\*
má drahá, dej mi víc své lásky, aúú\\*
Ja nechci skoro nic,\\*
já nechci skoro nic,\\*
já chci jen pohladit tvé vlásky, aúú

\end{poem}

\begin{poem}{Franky Dlouhán}{Brontosauři}

\settowidth{\versewidth}{kdo s ním chvilku byl, tak dlouho se pak smál.}

Kolik je smutného, když mraky černé jdou\\*
lidem nad hlavou, smutnou dálavou,\\*
já slyšel příběh, který velkou pravdu měl,\\*
za čas odletěl, každý zapomněl.

Ref.:\\*
Měl kapsu prázdnou Franky Dlouhán,\\*
po státech toulal se jen sám\\*
a že byl veselej, tak každej měl ho rád.\\*
Tam ruce k dílu mlčky přiloží\\*
a zase jede dál a každej,\\*
kdo s ním chvilku byl, tak dlouho se pak smál.

Tam, kde byl pláč,\\*
tam Frankey hezkou píseň měl,\\*
slzy neměl rád, chtěl se jenom smát,\\*
a když pak večer ranče tiše usínaj,\\*
Frankův zpěv jde dál, nocí s písní dál.

Ref.

Tak Frankyho vám jednou našli, přestal žít,\\*
jeho srdce spí, tiše smutně spí,\\*
bůhví jak a za co tenhle smíšek konec měl,\\*
farář píseň pěl, umíráček zněl.

Ref.

\end{poem}

\begin{poem}{Grónská písnička}{Jaromír Nohavica}

\settowidth{\versewidth}{neboť medvěd jim předvádí spoustu fíglů.}

Daleko na severu je Grónská zem,\\*
žije tam Eskymačka s Eskymákem.\\*
/:My bychom umrzli, jim není zima,\\*
snídají nanuky a eskyma.:/

Mají se bezvadně, vyspí se moc,\\*
půl roku trvá tam polární noc.\\*
/:Na jaře vzbudí se a vyběhnou ven,\\*
půl roku trvá tam polární den.:/

Když sněhu napadne nad kotníky,\\*
hrávají s medvědy na četníky.\\*
/:Medvědi těžko jsou k poražení,\\*
neboť medvědy ve sněhu vidět není.:/

Pokaždé ve středu, přesně ve dvě\\*
zaklepe na na íglů hlavní medvěd.\\*
/:"Dobrý den, mohu dál na vteřinu?\\*
Nesu vám trochu ryb na svačinu.":/

V kotlíku bublá čaj, kamna hřejí,\\*
psi venku hlídají před zloději.\\*
/:Smíchem se otřásá celé iglů,\\*
neboť medvěd jim předvádí spoustu fíglů.:/

Tak žijou vesele na severu,\\*
srandu si dělají z teploměrů.\\*
/:My bychom umrzli, jim není zima,\\*
neboť jsou doma a mezi svýma.:/
\end{poem}

\begin{poem}{Hlídač krav}{Jaromír Nohavica}

\settowidth{\versewidth}{a když je mi velmi smutno, lehnu do mokré trávy.}

Když jsem byl malý, říkali mi naši:\\*
Dobře se uč a jez chytrou kaši,\\*
až jednou vyrosteš, budeš doktorem práv.

Takový doktor si sedí pěkně v suchu,\\*
bere velký peníze a škrábe se v uchu.\\*
Já jim ale na to řek: Chci být hlídačem krav.

Já chci mít čapku s bambulí nahoře,\\*
jíst kaštany a mýt se v lavoře,\\*
od rána po celý den, zpívat si jen.\\*
Zpívat si:\\*
Pam pam pa dam pam padáda dam\\*
pam pam padam pam padádam\\*
pam padadadam padadadádam

K Vánocům mi kupovali hromady knih,\\*
co jsem ale vědět chtěl, to nevyčet jsem z nich,\\*
nikde jsem se nedozvěděl, jak se hlídají krávy.

Ptal jsem se starších a ptal jsem se všech,\\*
každý na mě hleděl jako na pytel blech,\\*
každý se mě opatrně tázal na moje zdraví.

Já chci mít čapku s bambulí nahoře,\\*
jíst kaštany a mýt se v lavoře,\\*
od rána po celý den, zpívat si jen.\\*
Zpívat si:\\*
Pam pam pa dam pam padáda dam\\*
pam pam padam pam padádam\\*
pam padadadam padadadádam

Teď už jsem starší a vím co vím,\\*
mnohé věci nemůžu a mnohé smím\\*
a když je mi velmi smutno, lehnu do mokré trávy.

S nohama křížem a rukama za hlavou,\\*
koukám nahoru na oblohu modravou,\\*
kde se mezi mraky honí moje strakaté krávy.

Já chci mít čapku s bambulí nahoře,\\*
jíst kaštany a mýt se v lavoře,\\*
od rána po celý den, zpívat si jen.\\*
Zpívat si:\\*
Pam pam pa dam pam padáda dam\\*
pam pam padam pam padádam\\*
pam padadadam padadadádam\\*
Zpívat si:\\*
Pam pam pa dam pam padáda dam\\*
pam pam padam pam padádam\\*
pam padadadam padadadádam

\end{poem}

\begin{poem}{Hoja hoj}{Karel Svoboda}

\settowidth{\versewidth}{zvol si nejlepší ze všech řemesel,}

Chceš-li na světe býti převesel,\\*
zvol si nejlepší ze všech řemesel,\\*
chceš-li okouzlit dívku nevinnou,\\*
staň se vojákem staň se hrdinou.

Ref.:\\*
Hoja hoj, hoja hoj,\\*
v králi máme zastání,\\*
hoja, hoj, hoja hoj,\\*
bůh nás zachrání,\\*
hoja hoj, hoja hoj,\\*
hmoždíře a palcáty,\\*
hoja hoj, hoja hoj,\\*
holky vokatý!

Rány na buben máš-li ve vínku,\\*
nesmíš zaváhat ani vteřinku,\\*
sláva až přilbu tvojí pozlatí,\\*
stal ses mužem tím co se neztratí.

Ref.

Ref.

\end{poem}

\begin{poem}{Hvězdář}{UDG}

\settowidth{\versewidth}{v bělostných šatech pro nemocné,}

/:Ztrácíš se před očima\\*
rosteš jen ve vlastním stínu.\\*
Každá další vina\\*
odkrývá moji vinu.:/

\begin{altverse}
Ve vínu dávno nic nehledám, \\*
(Ve vínu dávno nic nehledám) \\*
\end{altverse}
nehledám.\\*
\begin{altverse}
Ve vínu dávno nic nehledám, \\*
(Ve vínu dávno nic nehledám) \\*
nehledám.
\end{altverse}

/:Jak luna mizí s nocí\\*
v bělostných šatech pro nemocné,\\*
prosit je zvláštní pocit,\\*
jen, ať je den, noc ne.:/

\begin{altverse}
Od proseb dávno nic nečekám, \\*
(Od proseb dávno nic nečekám) \\*
\end{altverse}
nečekám.\\*
\begin{altverse}
Od proseb dávno nic nečekám, \\*
(Od proseb dávno nic nečekám) \\*
nečekám.
\end{altverse}

Na chodbách v bludných kruzích \\*
zářivka vyhasíná,\\*
a já ti do infuzí \\*
chci přilít trochu vína.

Na nebi jiných sluncí, \\*
jak se tam asi cítíš,\\*
s nebeskou interpunkcí, \\*
jiným tulákům svítíš.

\begin{altverse}
Ve vínu dávno nic nehledám, \\*
(Ve vínu dávno nic nehledám) \\*
\end{altverse}
nehledám.\\*
\begin{altverse}
Ve vínu dávno nic nehledám, \\*
(Ve vínu dávno nic nehledám) \\*
nehledám.
\end{altverse}

/:Jak luna mizí s nocí\\*
v bělostných šatech pro nemocné,\\*
prosit je zvláštní pocit,\\*
jen, ať je den, noc ne.:/

Obzor než klesne níž, \\*
je ráno a ty spíš.\\*
Od vlků odraná, \\*
hvězdáře Giordana.

Obzor než klesne níž, \\*
je ráno a ty spíš.\\*
Od vlků odraná, \\*
hvězdáře Giordana.

Obzor než klesne níž, \\*
je ráno a ty spíš.\\*
Od vlků odraná, \\*
hvězdáře Giordana.
\end{poem}

\begin{poem}{Já viděl divoké koně}{Jaromír Nohavica}

\settowidth{\versewidth}{Běželi běželi bez uzdy a sedla}

/:Já viděl divoké koně, \\*
běželi soumrakem.:/\\*
/:Vzduch těžký byl a divně voněl\\*
tabákem.:/

/:Běželi běželi bez uzdy a sedla\\*
krajinou řek a hor.:/\\*
/:Sper to čert jaká touha je to vedla\\*
za obzor.:/

/:Snad vesmír nad vesmírem, \\*
snad lístek na věčnost.:/\\*
/:Naše touho ještě neumírej, \\*
sil máme dost.:/

/:V nozdrách sládne zápach klisen \\*
na břehu jezera.:/\\*
/:Milování je divoká píseň \\*
večera.:/

/:Stébla trávy sklání hlavu, \\*
staví se do šiku.:/\\*
/:Král s dvořany přijíždí na popravu \\*
zbojníků.:/

/:Chtěl bych jak divoký kůň běžet běžet,\\* 
nemyslet na návrat.:/\\*
/:S koňskými handlíři vyrazit dveře, \\*
to bych rád.:/

\end{poem}

\begin{poem}{Jesse James}{Greenhorns}

\settowidth{\versewidth}{Běželi běželi bez uzdy a sedla}

Jesse James chlapík byl,\\*
hodně lidí vodpravil, \\*
vlaky přepadával rád,\\*
boháčům uměl brát,\\*
chudákùm dával zas,\\*
přál bych vám abyste ho mohli znát.

Jó, Jesse ženu svou\\*
tady nechal ubohou\\*
a tři děcka říkám vám,\\*
ale tenhle přítel hadí\\*
ten vám Jesse zradí,\\*
já vím tenkrát v noci prásk ho sám.

Jednou vám byla noc,\\*
měsíc tenkrát svítil moc,\\*
když tu vláček zůstal stát,\\*
kdekdo ví, že ten vlak\\*
přepad James kabrňák,\\*
čistě sám se svým bráchou akorát.

Jó, Jesse ženu svou\\*
tady nechal ubohou\\*
a tři děcka říkám vám,\\*
ale tenhle přítel hadí\\*
ten vám Jesse zradí,\\*
já vím tenkrát v noci prásk ho sám.

Jedenkrát Jesse James\\*
sedí doma za stolem\\*
a svým dětem vypráví.\\*
Robert Ford v nočních tmách\\*
připlíží se jako had,\\*
a on vám Jesse Jamese odpraví.

Jó, Jesse ženu svou\\*
tady nechal ubohou\\*
a tři děcka říkám vám,\\*
ale tenhle přítel hadí\\*
ten vám Jesse zradí,\\*
já vím tenkrát v noci prásk ho sám.

\end{poem}

\begin{poem}{Jižní kříž}{Jan Nedvěd}

\settowidth{\versewidth}{světem protloukal ses, jak ten život pádí,}

Spí Jižní kříž,\\*
jak říkali jsme hvězdám kdysi v mládí,\\*
to na studený zemi\\*
ještě uměli jsme milovat a spát.

A dál, však to znáš,\\*
světem protloukal ses, jak ten život pádí,\\*
dneska písničky třeba vod Červánků\\*
dojmou tě, jak vrátil bys' to rád.

Zase toulal by ses Foglarovým rájem\\*
a stavěl Bobří hráz,\\*
se smečkou vlků čekal na jaro,\\*
jak stejská se, až po zádech jde mráz.

Spí Jižní kříž,\\*
vidíš všechna místa, kde jsi někdy byl,\\*
to když, naplněnej smutkem,\\*
jsi plakal, plakal nebo snil.

Zase toulal by ses Foglarovým rájem\\*
a stavěl Bobří hráz,\\*
se smečkou vlků čekal na jaro,\\*
jak stejská se, až po zádech jde mráz.
\end{poem}

\begin{poem}{Když mě brali za vojáka}{Jaromír Nohavica}

\settowidth{\versewidth}{krásně jsem si zabulil-lil-lil-lil,}

Když mě brali za vojáka,\\*
stříhali mě dohola,\\*
vypadal jsem jako blbec,\\*
jak ti všichni dokola-la-la-la,\\*
jak ti všichni dokola.

Zavřeli mě do kasáren,\\*
začali mě učiti,\\*
jak mám správný voják býti,\\*
a svou zemi chrániti-ti-ti-ti,\\*
a svou zemi chrániti.

Na pokoji po večerce,\\*
ke zdi jsem se přitulil,\\*
vzpomněl jsem si na svou milou,\\*
krásně jsem si zabulil-lil-lil-lil,\\*
krásně jsem si zabulil.

Když přijela po půl roce,\\*
měl jsem zrovna zápal plic,\\*
po chodbě furt někdo chodil,\\*
tak nebylo z toho nic-nic-nic-nic,\\*
tak nebylo z toho nic.

Neplačte vy oči móje,\\*
ona za to nemohla, protože\\*
mladá holka lásku potřebuje,\\*
tak si k lásce pomohla-la-la-la,\\*
tak si k lásce pomohla.

Major nosí velkou hvězdu,\\*
před branou ho potkala,\\*
řek jí že má zrovna volný kvartýr,\\*
tak se sbalit nechala-la-la-la,\\*
tak se sbalit nechala.

Co je komu do vojáka,\\*
když ho holka zradila,\\*
nashledanou pane Fráňo Šrámku,\\*
písnička už skončila-la-la-la,\\*
jakpak se vám líbila-la-la-la,\\*
no nic moc extra nebyla.

\end{poem}

\begin{poem}{Když se zamiluje kůň}{Zdeněk Svěrák, Jaroslav Uhlíř}

\settowidth{\versewidth}{láskou hlubokou jak tůň}

Když se zamiluje kůň\\*
tam někde v pastvinách,\\*
láskou hlubokou jak tůň\\*
tam někde v pastvinách.\\*
Když se zamiluje kůň\\*
koňskou láskou,\\*
zpívejte písničku\\*
pro jeho klisničku,\\*
nechte ho jít.

Když se zamiluje kůň\\*
tam někde v pastvinách,\\*
láskou hlubokou jak tůň\\*
tam někde v pastvinách.\\*
Když se zamiluje kůň\\*
koňskou láskou,\\*
zpívejte písničku\\*
pro jeho klisničku,\\*
nechte ho jít.

Nejkrásnější zvíře,\\*
zvíře pro rytíře\\*
jmenuje se kůň,\\*
jmenuje se kůň,\\*
Važte si ho, lidé,\\*
ať nám jich pár zbyde,\\*
jmenuje se kůň,\\*
jmenuje se kůň,\\*
jmenuje se kůň.

Slečna s bílou lysinkou,\\*
tam někde v pastvinách,\\*
bude brzy maminkou,\\*
tam někde v pastvinách.\\*
Vždyť se zamiloval kůň\\*
koňskou láskou,\\*
hřívu si navlnil, aby ji oslnil\\*
a cválá k ní.

Nejkrásnější zvíře,\\*
zvíře pro rytíře\\*
jmenuje se kůň,\\*
jmenuje se kůň.\\*
Važte si ho, lidé,\\*
ať nám jich pár zbyde,\\*
jmenuje se kůň,\\*
jmenuje se kůň,\\*
jmenuje se kůň.

Když se zamiluje kůň\\*
tam někde v pastvinách,\\*
láskou hlubokou jak tůň\\*
tam někde v pastvinách.\\*
Když se zamiluje kůň\\*
koňskou láskou,\\*
zpívejte písničku\\*
pro jeho klisničku,\\*
nechte ho jít.\\*
Zpívejte písničku\\*
pro jeho klisničku,\\*
nechte ho jít.

\end{poem}

\begin{poem}{Kometa}{Jaromír Nohavica}

\settowidth{\versewidth}{a o všech lidech, co kdy žili na téhle planetě.}

Spatřil jsem kometu, oblohou letěla,\\*
chtěl jsem jí zazpívat, ona mi zmizela,\\*
zmizela jako laň u lesa v remízku,\\*
v očích mi zbylo jen pár žlutých penízků.

Penízky ukryl jsem do hlíny pod dubem,\\*
až příště přiletí, my už tu nebudem,\\*
my už tu nebudem, ach, pýcho marnivá,\\*
spatřil jsem kometu, chtěl jsem jí zazpívat.

O vodě, o trávě, o lese,\\*
o smrti, se kterou smířit nejde se,\\*
o lásce, o zradě, o světě\\*
a o všech lidech, co kdy žili na téhle planetě.

Na hvězdném nádraží cinkají vagóny,\\*
pan Kepler rozepsal nebeské zákony,\\*
hledal, až nalezl v hvězdářských triedrech\\*
tajemství, která teď neseme na bedrech.

Velká a odvěká tajemství přírody,\\*
že jenom z člověka člověk se narodí,\\*
že kořen s větvemi ve strom se spojuje\\*
a krev našich nadějí vesmírem putuje.

Na na na\\*
Na ná na na na na\\*
Na na na\\*
Na ná na na na na

Spatřil jsem kometu, byla jak reliéf\\*
zpod rukou umělce, který už nežije,\\*
šplhal jsem do nebe, chtěl jsem ji osahat,\\*
marnost mne vysvlékla celého donaha.

Jak socha Davida z bílého mramoru\\*
stál jsem a hleděl jsem, hleděl jsem nahoru,\\*
až příště přiletí, ach, pýcho marnivá,\\*
my už tu nebudem, ale jiný jí zazpívá.

O vodě, o trávě, o lese,\\*
o smrti, se kterou smířit nejde se,\\*
o lásce, o zradě, o světě,\\*
bude to písnička o nás a kometě...

\end{poem}

\begin{poem}{Kozel}{Jaromír Nohavica}

\settowidth{\versewidth}{a o všech lidech, co kdy žili na téhle planetě.}

/:Byl jeden pán,:/\\*
/:ten kozla měl,:/\\*
/:velice si:/\\*
/:s ním rozuměl.:/

/:Měl ho moc rád,:/\\*
/:opravdu moc.:/\\*
/:Hladil mu fous:/\\*
/:na dobrou noc.:/

/:Jednoho dne:/\\*
/:se kozel splet',:/\\*
/:rudé tričko:/\\*
/:pánovi sněd'.:/

/:Když to pán zřel,:/\\*
/:zařval jejé.:/\\*
/:Svázal kozla:/\\*
/:na koleje.:/

/:Zapískal vlak,:/\\*
/:kozel se lek'.:/\\*
/:To je má smrt,:/\\*
/:mečel mek mek.:/

/:Jak tak mečel,:/\\*
/:vykašlal pak:/\\*
/:rudé tričko,:/\\*
/:čímž stopnul vlak.:/

\end{poem}

\begin{poem}{Krysař}{Znouzectnost}

\settowidth{\versewidth}{a hodně dlouho v tomhle městě byl cítit vzduch křivdou.}

Bylo nebylo, kde se to tu vzalo,\\*
černé na bílém to na plakátech stálo.\\*
Bude tu hrát krysař a ty jeho krysy\\*
a bude to muzika dobrá jako kdysi.

Bylo nebylo, pak na scéně stáli,\\*
potkat je v noci ve městě, možná byste se báli.\\*
Středověký škorně a staletou halenu,\\*
do copánků spletený vlasy barvy havranů.

Ref.:\\*
Bylo nebylo, bylo nebylo,\\*
bylo nebylo a bylo nebylo.\\*
\tab A bylo. \\*
Bylo nebylo, bylo nebylo,\\*
bylo nebylo a bylo nebylo.

Bylo nebylo, kde se vlastně vzali\\*
a tou kouzelnou muzikou štěstí rozdávali.\\*
Pak ten okamžik skončil a nikdo nechtěl domů,\\*
atmosféra, kdo tam nebyl, nevěřil by tomu.

Ref.

Bylo nebylo, uplynul ňákej čas\\*
a plakáty na nárožích visely tu zas.\\*
Lidi zachvátila horečka a každý tam chtěl jít,\\*
bez rozdílu názorů zas to kouzlo prožít.

Ref.

Bylo nebylo, prošly divný zprávy,\\*
že vyšlehnou hranice pro čerty a ďábly.\\*
Do datumu na plakátech chybělo pár dní\\*
a na radnici Krysaře zvou si páni radní.

Ref.

Bylo nebylo, řekli, to by teda nešlo,\\*
s touhle vaší vizáží hrát budete tu těžko!\\*
Nás vůbec nezajímá, že vás mají lidi rádi,\\*
my jsme tady od toho jejich mysl chránit!

Ref.

Bylo nebylo, jako v tý pohádce,\\*
Krysař město opouští a hudba zní v dálce.\\*
Jak za kouzelnou píšťalou za ním myšlenky táhnou\\*
a hodně dlouho v tomhle městě byl cítit vzduch křivdou.

Ref.
\end{poem}

\begin{poem}{Little Big Horn}{Greenhorns}

\settowidth{\versewidth}{Tam, kde leží Little Big Horn, je indiánská zem,}

Tam, kde leží Little Big Horn, je indiánská zem,\\*
tam přijíždí generál Custer se svým praporem,\\*
modrý kabáty jezdců, stíny dlouhejch karabin,\\*
a z indiánskejch signálů po nebi letí dým.

Ref.:\\*
Říkal to Jim Bridger já měl jsem v noci sen,\\*
pod sedmou kavalerií jak krví rudne zem,\\*
kmen Siouxů je statečný a dobře svůj kraj zná,\\*
proč Custer neposlouchá ta slova varovná.

Tam blízko Little Big Hornu šedivou prérií\\*
táhne generál Custer se svou kavalerií,\\*
marně mu stopař Bridger radí: Zpátky povel dej!,\\*
Jedinou možnost ještě máš, život si zachovej!

Ref.

Tam blízko Little Big Hornu se vznáší smrti stín,\\*
padají jezdci z koní, výstřely z karabin,\\*
límce modrejch kabátů barví krev červená,\\*
kmen Siouxů je statečný a dobře svůj kraj zná.

Ref.

Pak všechno ztichlo a jen tamtam duní nad krajem,\\*
v oblacích prachu mizí Siouxů vítězný kmen,\\*
cáry vlajky hvězdnatý po kopcích vítr vál,\\*
tam uprostřed svých vojáků leží i generál.

Ref.

\end{poem}

\begin{poem}{Malování}{Divokej Bill}

\settowidth{\versewidth}{jen táta a máma jsou s náma,}

Nesnaž se,\\*
znáš se.\\*
Řekni mi, co je jiný,\\*
jak v kleci máš se\\*
pro nevinný,\\*
noci dlouhý\\*
jsou plný touhy\\*
a lásky nás dvou.

Všechno hezký za sebou mám,\\*
můžu si za to sám.\\*
V hlavě hlavolam,\\*
jen táta a máma\\*
jsou s náma,\\*
napořád s náma.

To je to tvoje malování\\*
vzdušnejch zámků,\\*
malování\\*
po zdech holejma rukama\\*
tě nezachrání,\\*
už máš na kahánku,\\*
tě nezachrání,\\*
už seš na zádech.

\vfill\eject

Je to za náma,\\*
ty čteš poslední stránku,\\*
za náma,\\*
na zádech,\\*
za náma,\\*
už máš na kahánku,\\*
mezi náma,\\*
mi taky došel dech.

To je to tvoje malování\\*
vzdušnejch zámků,\\*
malování\\*
po zdech holejma rukama,\\*
tě nezachrání,\\*
už máš na kahánku,\\*
tě nezachrání,\\*
už seš na zádech.

Je to za náma,\\*
ty čteš poslední stránku,\\*
za náma,\\*
na zádech,\\*
za náma,\\*
už máš na kahánku,\\*
mezi náma,\\*
mi taky došel dech.

Znáš se.\\*
Řekni mi, co je jiný,\\*
jak v kleci máš se\\*
pro nevinný,\\*
noci dlouhý\\*
jsou plný touhy\\*
a lásky nás tří.

\end{poem}

\begin{poem}{Maruška}{Malomocnost prázdnoty}

\settowidth{\versewidth}{Co jste to za vojsko,}

/:Na malém plácku:/\\*
/:děti si hrají,:/\\*
/:hrají si na válku,:/\\*
/:všechno už mají.:/

/:Stejnokroj z tepláků:/\\*
/:větší než na míru,:/\\*
/:dřevěný pistole,:/\\*
/:čepice z papíru.:/

/:A bitva za bitvou,:/\\*
/:tak to jde dokola,:/\\*
/:až potom najednou:/\\*
/:něčí hlas zavolá::/

Ref.:\\*
/:Zastavte válku,:/\\*
/:Maruška brečí,:/\\*
/:dostala kamenem:/\\*
/:při naší zteči.:/

/:Co jste to za vojsko,:/\\*
/:když místo střílení:/\\*
/:do svých nepřátel:/\\*
/:házíte kamení.:/

/:Na to my nehrajem,:/\\*
/:vy nám to kazíte,:/\\*
/:nedbáte pravidel,:/\\*
/:A pak se divíte!:/

/:Na velkém plácku:/\\*
/:hrají si dospělí,:/\\*
/:jen místo dřevěných:/\\*
/:hračky maj z oceli.:/

/:Ocel se zarývá:/\\*
/:do kůry stromů,:/\\*
/:desítky jizviček:/\\*
/:a blesky hromů.:/

/:Některým na duši,:/\\*
/:některým do těla:/\\*
/:vpálí znamení,:/\\*
/:proč nikdo nevolá?:/

Ref.

\end{poem}

\begin{poem}{Mezi horami}{Čechomor}

\settowidth{\versewidth}{hned na hrob padla a viac něvstala}

/:Mezi horami\\*
lipka zelená,:/\\*
/:zabili Janka, Janíčka, Janka\\*
miesto jeleňa.:/

/:Keď ho zabili,\\*
zamordovali,:/\\*
/:na jeho hrobě, na jeho hrobě\\*
kříž postavili.:/

/:Ej křížu, křížu\\*
ukřižovaný,:/\\*
/:zde leží Janík, Janíček, Janík,\\*
zamordovaný.:/

/:Tu šla Anička\\*
plakat Janíčka,:/\\*
/:hned na hrob padla a viac něvstala\\*
dobrá Anička.:/

Mezi horami... 
\end{poem}

\begin{poem}{Měsíc}{Mňága a Žďorp}

\settowidth{\versewidth}{teďka svítí měsíc pro každýho zvlášť mně je to líto}

Děkuju ti za to, žes mi ještě zavolala\\*
moje duše černá už to ani nečekala.\\*
Jenže moje drahá, je to všechno trochu na nic\\*
já už jsem se rozhod - zítra budu znovu panic. 

Děkuju ti za to, žes to rovnou nepoložil,\\*
že jsi nezapomněl, co jsi se mnou všechno prožil.\\*
Teďka svítí měsíc pro každýho zvlášť mně je to líto\\*
jenže už je konec však víš to...\\*
...vím to!

měsíc\\*
svítí měsíc

měsíc\\*
svítí měsíc

měsíc\\*
svítí měsíc

měsíc\\*
svítí měsíc

\end{poem}

\begin{poem}{Milenci v texaskách}{Josef Zíma, Karel Štědrý}

\settowidth{\versewidth}{Tak jeden mladík s jednou slečnou}

Chodili spolu z čisté lásky\\*
a sedmnáct jim bylo let\\*
a do té lásky bez nadsázky\\*
se vešel celý širý svět.

Ten svět v nich ale viděl pásky\\*
a jak by mohl nevidět,\\*
vždyť horovali pro texasky\\*
a sedmnáct jim bylo let.

A v jedné zvláště slabé chvíli,\\*
za noci silných úkladů,\\*
ti dva se spolu oženili\\*
bez požehnání úřadů.

Ať vám to je, či není milé,\\*
měla ho ráda, měl ji rád.\\*
Odpusťte dívce provinilé,\\*
jestli vám o to bude stát.

Ať vám to je, či není milé,\\*
měla ho ráda, měl ji rád\\*
a bylo by moc pošetilé\\*
pro život hledat jízdní řád.

Tak jeden mladík s jednou slečnou\\*
se spolu octli na trati.\\*
Kéž dojedou až na konečnou,\\*
kéž na trati se neztratí,\\*
kéž na trati se neztratí,\\*
kéž na trati se neztratí.

\end{poem}

\begin{poem}{Montgomery}{Mirek Skunk Jaroš}

\settowidth{\versewidth}{déšť mu slepil vlasy jako jíl.}

Déšť ti, holka, smáčel vlasy,\\*
z tvých očí zbyl prázdný kruh.\\*
Kde je zbytek tvojí krásy,\\*
to ví dneska snad jenom Bůh.

Ref.:\\*
Z celé Jižní eskadrony\\*
nezbyl ani jeden muž.\\*
V Montgomery bijou zvony,\\*
déšť ti smejvá ze rtů růž.

Tam na kopci v prachu cesty\\*
leží i tvůj generál.\\*
V ruce šátek od nevěsty,\\*
ale ruka leží dál.

Ref.

Tvář má zšedivělou strachem,\\*
zbylo v ní pár těžkých chvil.\\*
Proužek krve stéká prachem,\\*
déšť mu slepil vlasy jako jíl.

Ref.

Déšť ti šeptá jeho jméno,\\*
šeptá ho i listoví.\\*
Lásku měl rád víc než život,\\*
to ti nikdy nepoví.

Ref.

\end{poem}

\begin{poem}{Obluda}{Hop Trop}

\settowidth{\versewidth}{křik' plavčík na stožáru, a hlásek se mu třás'}

Jó, dvěstě nás tam bylo na brize do Číny\\*
a furt se jenom pilo, až tuhly ledviny,\\*
a když už bylo k ránu a všichni pod vobraz,\\*
křik' plavčík na stožáru, a hlásek se mu třás':

Já tady nebudu, já vidím vobludu,\\*
já vocaď pryč pudu, jímá mě strach,\\*
chyťte tu vobludu, sežere palubu,\\*
tohle je vo hubu, ach, ich, och, ach.

Má vocas dvěstě sáhů a ploutev přes hektar\\*
a zubů plnou hubu, a rozcvičuje spár\\*
a rozčileně mrká svým vokem jediným,\\*
a páchne, řve a krká a z tlamy pouští dým.

Já tady nebudu, já vidím vobludu,\\*
já vocaď pryč pudu, jímá mě strach,\\*
chyťte tu vobludu, sežere palubu,\\*
tohle je vo hubu, ach, ich, och, ach.

Rum kapitánem hází, chce pálit z kanónu,\\*
zapomněl, že ho v září prochlastal v Kantonu,\\*
když uviděl tu bídu, hned zpotil se jak myš,\\*
poklekl k komínu, pěl: K tobě, Bože, blíž.

Já tady nebudu, já vidím vobludu,\\*
já vocaď pryč pudu, jímá mě strach,\\*
chyťte tu vobludu, sežere palubu,\\*
tohle je vo hubu, ach, ich, och, ach.

Náš kormidelník chrabrý je drsná povaha,\\*
zalehl u zábradlí a házel flaškama,\\*
když na vobludě bouchl kanystr vod ginu,\\*
tu přízrak vztekle houkl a vlítl na brigu.

Já tady nebudu, já vidím vobludu,\\*
já vocaď pryč pudu, jímá mě strach,\\*
chyťte tu vobludu, sežere palubu,\\*
tohle je vo hubu, ach, ich, och, ach.

Tři stěžně rozlámala na naší kocábce,\\*
však ztuhla jako skála, když čuchla k posádce,\\*
pak démon alkoholu ji srazil v oceán,\\*
my slezli zase dolů a chlastali jsme dál.

Já tady nebudu, já vidím vobludu,\\*
já vocaď pryč pudu, jímá mě strach,\\*
chyťte tu vobludu, sežere palubu,\\*
tohle je vo hubu, ach, ich, och, ach.\\*
ach, ich, och, ach ...

\end{poem}

\begin{poem}{Píseň zhrzeného trampa}{Jaromír Nohavica}

\settowidth{\versewidth}{Že prý se můžu vrátit zpět až dám se do cajku}

\begin{altverse}
Poněvadž nemám kanady
a neznám písně z pamp\\*
(johoho a neznám písně z pamp)\\*
vyloučili mě z osady
že prý jsem houby tramp\\*
(johoho že prý jsem houby tramp)
\end{altverse}

\begin{altverse}
Napsali si do cancáků
jen ať to každý ví\\*
(johoho jen ať to každý ví)\\*
Vyloučený z řad čundráků
ten frajer libový\\*
(johoho ten frajer libový)
\end{altverse}

Ref.:\\*
Já jsem ostuda traperů\\*
já mám rád operu\\*
já mám rád jazz-rock\\*
chodím po světě bez nože\\*
to prý se nemože\\*
to prý jsem cvok\\*
já jsem nikdy neplul na šífu\\*
a všem šerifům jsem říkal Ba ne\\*
pane\\*
já jsem ostuda trempů\\*
já když chlempu\\*
tak v autokempu

\begin{altverse}
Povídal mi frajer Joe
jen žádný legrácky\\*
(johoho jen žádný legrácky)\\*
jinak chytneš na bendžo
čestný čundrácký\\*
(johoho a čestný čundrácký)
\end{altverse}

\begin{altverse}
Že prý se můžu vrátit zpět
až dám se do cajku\\*
(johoho až se dám do cajku)\\*
a odříkám jim nazpaměť
akordy na Vlajku\\*
(johoho akordy na Vlajku)
\end{altverse}

Ref.

\begin{altverse}
A tak chodím po světě
a mám zaracha\\*
(johoho a mám zaracha)\\*
na vandr chodím k Markétě
a dávám si bacha\\*
(johoho a dávam si bacha)
\end{altverse}

\begin{altverse}
Dokud se trampské úřady
nepoučí z chyb\\*
(johoho a nepoučí z chyb)\\*
zpívám si to svý nevadí
a zase bude líp\\*
(johoho a zase bude líp)
\end{altverse}

Ref.

\end{poem}

\begin{poem}{Pochod marodů}{Jaromír Nohavica}

\settowidth{\versewidth}{ale už jsou nadranc, -dranc, -dranc,}

Krabička cigaret\\*
a do kafe rum, rum, rum,\\*
dvě vodky a Fernet\\*
a teď, doktore, čum, čum, čum,\\*
chrapot v hrudním koši,\\*
no to je zážitek,\\*
my jsme kámoši\\*
řidičů sanitek, -tek, -tek.

Měli jsme ledviny,\\*
ale už jsou nadranc, -dranc, -dranc,\\*
i tělní dutiny\\*
už ztratily glanc, glanc, glanc,\\*
u srdce divný zvuk,\\*
co je to, nemám šajn,\\*
a je to vlastně fuk,\\*
žijem fajn, žijem fajn, fajn, fajn.

Cirhóza, trombóza,\\*
dávivý kašel,\\*
tuberkulóza\\*
- jó, to je naše!\\*
neuróza, skleróza,\\*
ohnutá záda,\\*
paradentóza,\\*
no to je paráda!\\*
Jsme slabí na těle,\\*
ale silní na duchu,\\*
žijem vesele,\\*
juchuchuchuchu!

Už kolem nás chodí\\*
pepka mrtvice, -ce, -ce,\\*
tak pozor, marodi,\\*
je zlá velice, -ce, -ce,\\*
zná naše adresy\\*
a je to čiperka,\\*
koho chce, najde si,\\*
ten natáhne perka, -rka, -rka.

Zítra nás odvezou,\\*
bude veselo, -lo, -lo,\\*
medici vylezou\\*
na naše tělo, -lo, -lo,\\*
budou nám řezati\\*
ty naše vnitřnosti\\*
a přitom zpívati\\*
ze samé radosti, -sti, -sti.

Zpívati: cirhóza, trombóza,\\*
dávivý kašel,\\*
tuberkulóza,\\*
hele, já jsem to našel!\\*
Neuróza, skleróza,\\*
křivičná záda,\\*
paradentóza,\\*
no to je paráda!\\*
Byli slabí na těle,\\*
ale silní na duchu,\\*
žili vesele,\\*
než měli poru\\*
-chu -chu -chu -chu -chu -chu\\*
-chů -chu -chu -chu -chu\\*
-chu -chu -chu -chů -chu\\*
-chu -chu -chu -chu.

\end{poem}

\begin{poem}{Proklínám}{Janek Ledecký}

\settowidth{\versewidth}{Sám s hlavou skloněnou, všechny lásky budou zdání.}

Prázdnej byt je jako past, kde růže uvadnou.\\*
Potisící čtu Tvůj dopis na rozloučenou.\\*
Píšeš, že odcházíš, když den se s nocí střídá,\\*
vodu z vína udělá, kdo dobře nehlídá.

Píšeš:\\*
Proklínám, ty Tvoje ústa proklínám,\\*
Tvoje oči ledový, v srdci jen sníh.\\*
Sám a sám, ať nikdy úsvit nespatříš,\\*
na ústa mříž, oči oslepnou, ať do smrti seš sám.

Tvoje oči jsou jak stín a tvář den když se stmívá,\\*
stromy rostou čím dál výš a pak je čeká pád.\\*
Sám s hlavou skloněnou, všechny lásky budou zdání.\\*
Potisící čtu Tvůj dopis na rozloučenou.

Píšeš:\\*
Proklínám, ty Tvoje ústa proklínám,\\*
Tvoje oči ledový, v srdci jen sníh.\\*
Sám a sám, ať nikdy úsvit nespatříš,\\*
na ústa mříž, oči oslepnou, ať do smrti seš sám.

Sám a sám, ať nikdy úsvit nespatříš,\\*
na ústa mříž, oči oslepnou, ať do smrti seš sám.

\end{poem}

\begin{poem}{Proměny}{Čechomor}

\settowidth{\versewidth}{A ty přece budeš má, lebo mi tě pán Bůh dá.}

Darmo sa ty trápíš, můj milý synečku,\\*
nenosím já tebe, nenosím v srdéčku.\\*
Přece tvoja nebudu, ani jednu hodinu.

Copak sobě myslíš, má milá panenko,\\*
vždyť ty jsi to moje rozmilé srdénko.\\*
A ty musíš býti má, lebo mi tě pán Bůh dá.

A já sa udělám malú veveričkú,\\*
a uskočím tobě z dubu na jedličku.\\*
Přece tvoja nebudu, ani jednu hodinu.

A já chovám doma takú sekerečku,\\*
ona mi podetne důbek i jedličku.\\*
A ty musíš býti má, lebo mi tě pán Bůh dá.

A já sa udělám tu malú rybičkú,\\*
a já ti uplynu pryč po Dunajíčku.\\*
Přece tvoja nebudu, ani jednu hodinu.

A já chovám doma takovú udičku,\\*
co na ni ulovím kdejakú rybičku.\\*
A ty přece budeš má, lebo mi tě pán Bůh dá.

A já sa udělám tú velikú vranú\\*
a já ti uletím na uherskú stranu.\\*
Přece tvoja nebudu, ani jednu hodinu.

A já chovám doma starodávnú kušu,\\*
co ona vystřelí všeckým vranám dušu.\\*
A ty musíš býti má, lebo mi tě pán Bůh dá.

A já sa udělám hvězdičkú na nebi\\*
a já budu lidem svítiti na zemi.\\*
Přece tvoja nebudu, ani jednu hodinu.

A sú u nás doma takoví hvězdáři,\\*
co vypočítajú hvězdičky na nebi.\\*
A ty musíš býti má, lebo mi tě pán Bůh dá.

A ty musíš býti má, lebo mi tě pán Bůh dá.

\end{poem}

\begin{poem}{Severní vítr}{Zdeněk Svěrák, Jaroslav Uhlíř}

\settowidth{\versewidth}{svůj hrob, a že stloukám si kříž.}

Jdu s děravou patou,\\*
mám horečku zlatou,\\*
jsem chudý, jsem sláb, nemocen.\\*
Hlava mě pálí\\*
a v modravé dáli\\*
se leskne a třpytí můj sen.

Kraj pod sněhem mlčí,\\*
tam stopy jsou vlčí,\\*
tam zbytečně budeš mi psát.\\*
Sám v dřevěné boudě\\*
sen o zlaté hroudě\\*
já nechám si tisíckrát zdát.

Severní vítr je krutý,\\*
počítej lásko má s tím.\\*
K nohám ti dám zlaté pruty\\*
nebo se vůbec nevrátím.

Tak zarůstám vousem\\*
a vlci už jdou sem,\\*
už slyším je výt blíž a blíž.\\*
Už mají mou stopu,\\*
už větří, že kopu\\*
svůj hrob, a že stloukám si kříž.

Zde leží ten blázen,\\*
chtěl dům a chtěl bazén\\*
a opustil tvou krásnou tvář.\\*
Má plechovej hrnek\\*
a pár zlatejch zrnek\\*
a nad hrobem polární zář.

Severní vítr je krutý,\\*
počítej lásko má s tím.\\*
K nohám ti dám zlaté pruty\\*
nebo se vůbec nevrátím.

\end{poem}

\begin{poem}{Svařák}{Harlej}

\settowidth{\versewidth}{svůj hrob, a že stloukám si kříž.}

/:Když jsem sám doma,:/\\*
/:poslouchám Vávra,:/\\*
/:starýho vola,:/\\*
/:pořád dokola.:/

/:Chce to mít nápad,:/\\*
/:a ne pořád chrápat,:/\\*
/:já dostal jsem nápad,:/\\*
/:udělat mejdan.:/

Mám rád svařené víno červené,\\*
já mám rád rád svařák.\\*
Mám rád svařené víno červené,\\*
já mám rád rád svařák.

/:Se známou partou,:/\\*
/:domluva krátká,:/\\*
/:zejtra v šest hodin,:/\\*
/:vstup jedna lampa.:/

/:Začíná mejdan,:/\\*
/:na 200 procent,:/\\*
/:my plníme plány,:/\\*
/:rostou nám blány.:/

Mám rád svařené víno červené,\\*
já mám rád rád svařák.\\*
Mám rád svařené víno červené,\\*
já mám rád rád svařák.

\end{poem}

\begin{poem}{Špinavej Titanik}{Malomocnost prázdnoty}

\settowidth{\versewidth}{snad ke štěstí, snad do záhuby}

Všude kolem prázdno,\\*
loď pluje v úplném bezvětří,\\*
v podpalubí bahno,\\*
na palubě špína.\\*
Čert ví, kam pluje,\\*
snad ke štěstí, snad do záhuby,\\*
na kupách hnoje\\*
smrtka se na ně zubí.

Opilý námořník\\*
s opilou děvkou\\*
na lodi souloží\\*
pod mokrou plachtou.\\*
Nic kolem nevidí,\\*
nic nevnímají,\\*
jenom se houpají\\*
na vlnách chtíče.

Na týhletý lodi\\*
s kormidlem v půli zlomeným\\*
cestující chodí\\*
tváří se štastně, jsou veselí!\\*
Všichni jsou slepí,\\*
svůj osud radši netuší,\\*
nevidí útesy\\*
nasáklý smrtí.

\vfill\eject

Opilý námořník\\*
s opilou děvkou\\*
na lodi souloží\\*
pod mokrou plachtou.\\*
Nic kolem nevidí,\\*
nic nevnímají,\\*
jenom se houpají\\*
na vlnách chtíče.

Trosky lodi plavou\\*
po mořské hladině v bezvětří,\\*
moře plné lidí,\\*
na břeh nikoho už nepustí!\\*
Obrovské bohatství,\\*
sobecky nastřádané,\\*
nikdo už nevlastní,\\*
válí se všude po dně,\\*
jen...

Opilý námořník\\*
s opilou děvkou\\*
na lodi souloží\\*
pod mokrou plachtou.\\*
Nic kolem nevidí,\\*
nic nevnímají,\\*
jenom se houpají\\*
na vlnách chtíče.

\end{poem}

\begin{poem}{Tři čuníci}{Jaromír Nohavica}

\settowidth{\versewidth}{vyšli prostě do světa a vesele si zpívají:}

V řadě za sebou tři čuníci jdou,\\*
ťápají si v blátě cestou-necestou,\\*
kufry nemají, cestu neznají,\\*
vyšli prostě do světa a vesele si zpívají: 

Ref.:\\*
Ui ui ui ui uí\\*
ui ui ui ui uí\\*
ui ui ui ui uí uí\\*
ui ui ui ui uí

Ui ui ui ui uí\\*
ui ui ui ui uí\\*
ui ui ui ui ui ui ui ui\\*
ui ui ui ui ui ui uí

Auta jezdí tam, náklaďáky sem,\\*
tři čuníci jdou, jdou rovnou za nosem,\\*
ušima bimbají, žito křoupají,\\*
vyšli prostě do světa a vesele si zpívají: 

Ref.

Levá, pravá - teď!, přední, zadní - už!,\\*
tři čuníci jdou, jdou jako jeden muž,\\*
lidé zírají, důvod neznají,\\*
proč ti malí čuníci tak vesele si zpívají: 

Ref.

Když kopýtka pálí, když jim dojde dech,\\*
sednou ke studánce na vysoký břeh,\\*
ušima bimbají, kopýtka máchají,\\*
chvilinku si odpočinou, a pak dál se vydají: 

Ref.

Když se spustí déšť, roztrhne se mrak,\\*
k sobě přitisknou se čumák na čumák,\\*
blesky bleskají, kapky pleskají,\\*
oni v dešti, nepohodě vesele si zpívají: 

Ref.

Za tu spoustu let, co je světem svět,\\*
přešli zeměkouli třikrát tam a zpět\\*
v řadě za sebou, hele, támhle jdou,\\*
pojďme s nima zazpívat si jejich píseň veselou: 

Ref.

\end{poem}

\begin{poem}{Tři kříže}{Hop trop}

\settowidth{\versewidth}{ale odpouštět božský, snad mi tedy Bůh odpustí."}

Dávám sbohem břehům proklatejm,\\*
který v drápech má ďábel sám.\\*
Bílou přídí šalupa My Grave\\*
míří k útesům, který znám.

Ref.:\\*
Jen tři kříže z bílýho kamení\\*
někdo do písku poskládal.\\*
Slzy v očích měl a v ruce znavený\\*
lodní deník, co sám do něj psal.

První kříž má pod sebou jen hřích,\\*
samý pití a rvačky jen,\\*
chřestot nožů, při kterým přejde smích,\\*
srdce kámen a jméno Sten.

Ref.

Já Bob Green mám tváře zjizvený,\\*
štěkot psa zněl, když jsem se smál.\\*
Druhej kříž mám a spím pod zemí,\\*
že jsem falešný karty hrál.

Ref.

Třetí kříž snad vyvolá jen vztek,\\*
Davy Rodgers těm dvoum život vzal\\*
Svědomí měl a vedle nich si klek:\\*
"Vím, trestat je lidský,\\*
ale odpouštět božský, snad mi tedy Bůh odpustí."

Ref.

\end{poem}

\begin{poem}{Titanic}{Tři sestry}

\settowidth{\versewidth}{Slavný muž, jenž měl z pekla štěstí}

Z výšin Velkých Karpat\\*
Přes moře na západ.\\*
Bída v Americe má cíl.\\*
Ráno maj už lístek.\\*
Titanic je místem,\\*
Kde teď budou týdnů pár žít.

Jen Ján zpitej zas jak Dán.\\*
Prošvihnul odjezd parníku snů.\\*
Máma ta pláče že sám\\*
Odsoudil se skrz pitku\\*
Ke smutnému zítřku svých dnů.

Přístavem se toulá\\*
A lístek svůj žmoulá.\\*
Protáhnul ten mejdan pár dní.\\*
Náhle z novin zpráva,\\*
Že ledovec jak kráva\\*
Potopil ten slavný parník.

Pán Ján zpitej zas jak Dá.\\*
Slavný muž, jenž měl z pekla štěstí.\\*
Vrátí se zpátky jen sám.\\*
Odchází z Karpat bída.\\*
On hodlá dál žít na scestí.

\end{poem}

\begin{poem}{Udavač z Těšína}{Jaroslav Hutka}

\settowidth{\versewidth}{vtíravým hlasem svým bude nám zpívat}

Udavač z Těšína, koncert mu začíná\\*
vtíravým hlasem svým bude nám zpívat\\*
v sále se zhasíná, dám si sklenku vína\\*
s publikem budu se na něj též dívat

Jak je ten udavač krásný\\*
jak velký je umělec\\*
výraz má zřetelně jasný\\*
dnes nám zpívá zbabělec

Zasněné obrazy, láska z nich vychází\\*
básnické obraty, půvabné rýmy\\*
Dívky jsou raněné, ženy jsou zmámené\\*
a muži ztrácejí pod sebou zem

Jak je ten udavač krásný\\*
jak velký je umělec\\*
výraz má zřetelně jasný\\*
dnes nám zpívá zbabělec

Kritik si notuje, na písních hoduje\\*
sál zpívá chytlavý, radostný refrén\\*
Poslouchám pozorně, už je mi odporně\\*
odněkud slyším zpěv homérských Sirén

Jak je ten udavač krásný\\*
jak velký je umělec\\*
výraz má zřetelně jasný\\*
dnes nám zpívá zbabělec

Na cestě do Vídně zachoval se bídně\\*
písničkáře Kryla jidášsky objal\\*
Klíče mu věnoval, pak ho fízlům udal\\*
zatlouká zatlouká a tím mě dojal

Jak je ten udavač krásný\\*
jak velký je umělec\\*
výraz má zřetelně jasný\\*
dnes nám zpívá zbabělec

Tam za bolševiků zpíval všem morálku\\*
hrdina před všemi prvního řádu\\*
Kritikou oceněn, anděly odměněn\\*
bohatě co blíží se ke svému pádu

\end{poem}

\begin{poem}{Vlak v 05}{Greenhorns}

\settowidth{\versewidth}{v němž jede parta, se kterou jsem chodil pít:}

Já ti řeknu, proč jsem dneska divnej tak,\\*
kdy' v dálce za tunelem začne houkat vlak,\\*
proč nechávám svou ciaretu vyhasnout,\\*
proč každou chvíli stojím jako solnej sloup.

Touhle tratí já jsem jezdil řadu let.\\*
Mašina a koleje byly můj svět\\*
a parta z vlaku, se kterou jsem chodil pít:\\*
strojvůdce Mike, výhybkář Joe a brzdař Kid.

Jednou, když jsme jeli z Friska v 05,\\*
tahle trať nebejvala žádnej med,\\*
po banánový slupce sklouz a z vlaku slít\\*
a pod kolama zůstal mladej Kid.

Za rok nato z Friska 05\\*
vyjel si Mike mašinfíra naposled.\\*
Tam v zatáčce za tunelem při srážce\\*
z budky vylít a zlomil si vaz o pražce.

Pak jsem už zbyl jen já a Výhybka Joe,\\*
ale neštěstí chodilo kolem nás dvou.\\*
Jednou Výhybka Joe nevrátil se zpět,\\*
ani on ani ten vlak v 05.

Takže už víš, proč jsem teda bledej tak,\\*
já slyším za tunelem houkat divnej vlak,\\*
v němž jede parta, se kterou jsem chodil pít:\\*
strojvůdce Mike, Výhybka Joe a brzdař Kid.

Ten vlak se ke mně řítí tmou,\\*
tak ahoj Kide, Miku, ahoj Joe...

\end{poem}

\begin{poem}{Zajíci}{Jaromír Nohavica}

\settowidth{\versewidth}{až to tu skončí, sraz je v sedm večer na mezi}

Podzimní bílá mlha válí se na mechu,\\*
myslivci vstali už ráno,\\*
zajíci zalezli a nechce se jim z pelechu,\\*
neboť se chtějí dožít Vánoc.

Ref.:\\*
Vzduchem zní famfára,\\*
tram-ta-rá ra-ra-ra,\\*
po poli kráčejí střelci,\\*
já se svou kytarou,\\*
zpívám písničku prastarou\\*
o tom že malí budou velcí.

Zajíci naposledy potřásli si tlapkama\\*
a potom oblékli dresy,\\*
začíná specielní slalom mezi brokama,\\*
na trati pole-louky-lesy.

Ref.

Už volá vrchní zajíc Pravda vítězí!\\*
nohy jsou naše hlavní zbraně,\\*
až to tu skončí, sraz je v sedm večer na mezi\\*
a teď zajíci hurá na ně!

Ref.

\end{poem}

\begin{poem}{Zatanči}{Jaromír Nohavica}

\settowidth{\versewidth}{ať tvůj šat má milá ať tvůj šat na zemi skončí}

Zatanči má milá zatanči pro mé oči,\\*
zatanči a vetkni nůž do mých zad,\\*
ať tvůj šat má milá ať tvůj šat na zemi skončí,\\*
ať tvůj šat má milá rázem je sňat.

Ref.:\\*
Zatanči jako se okolo ohně tančí,\\*
zatanči jako na vodě loď,\\*
zatanči jako to slunce mezi pomeranči,\\*
zatanči a pak ke mně pojď.

Polož dlaň má milá polož dlaň na má prsa,\\*
polož dlaň nestoudně na moji hruď,\\*
obejmi má milá obejmi moje bedra,\\*
obejmi je pevně a mojí buď.

Ref.

Nový den než začne má milá nežli začne,\\*
nový den než začne nasyť můj hlad,\\*
zatanči má milá pro moje oči lačné,\\*
zatanči a já budu ti hrát.

Ref.

Ref.

\end{poem}

\begin{poem}{Zatracenej život}{Greenhorns}

\settowidth{\versewidth}{než jsem jí stačil řict, že je hezká,}

To bylo v Dakotě po vejplatě,\\*
whisky jsem tam pašoval,\\*
a že jsem byl sám jako kůl v plotě,\\*
s holkou jsem tam špásoval.

Šel jsem s ní nocí, jak vede stezka,\\*
okolo Červených skal,\\*
než jsem jí stačil řict, že je hezká,\\*
zpěněnej býk se k nám hnal.

Povídám:\\*
Jupí, čerte, jdi radši dál,\\*
pak jsem ho za rohy vzal,\\*
udělal přemet a jako tur řval,\\*
do dáli upaloval.

To bylo v Dawsonu tam v saloonu\\*
a já jsem zase přebral,\\*
všechny svý prachy jsem měl v talónu,\\*
na život jsem nadával.

Zatracenej život, čert by to spral,\\*
do nebe jsem se rouhal,\\*
než jsem se u baru vzpamatoval,\\*
Belzebub vedle mě stál.

Povídám:\\*
Jupí, čerte, jdi radši dál,\\*
pak jsem ho za rohy vzal,\\*
udělal přemet a jako tur řval,\\*
do dáli upaloval.

Jó, rychle oplácí tenhleten svět,\\*
než bys napočítal pět,\\*
Ďáblovým kaňónem musel jsem jet,\\*
když jsem se vracel nazpět.

Jak se tak kolíbám, uzdu v pěsti,\\*
schylovalo se k dešti,\\*
Belzebub s partou stál vprostřed cesty,\\*
zavětřil jsem neštěstí.

Povídá:\\*
Jupí, čerte, jdi radši dál,\\*
potom mě za nohy vzal,\\*
udělal jsem přemet, jako tur řval,\\*
do dáli jsem upaloval.

\end{poem}

\begin{poem}{Zítra ráno v pět}{Jaromír Nohavica}

\settowidth{\versewidth}{že mně se nechce do nebes,}

Až mě zítra ráno v pět\\*
ke zdi postaví,\\*
ještě si naposled\\*
dám vodku na zdraví,\\*
z očí pásku strhnu si,\\*
to abych viděl na nebe,\\*
a pak vzpomenu si\\*
lásko na tebe.\\*
A pak vzpomenu si\\*
na tebe.

Až zítra ráno v pět\\*
přijde ke mně kněz,\\*
řeknu mu že se splet,\\*
že mně se nechce do nebes,\\*
že žil jsem jak jsem žil\\*
a stejně tak i dožiju,\\*
a co jsem si nadrobil\\*
to si i vypiju.\\*
A co jsem si nadrobil\\*
si i vypiju.

\vfill\eject

Až zítra ráno v pět\\*
poručík řekne pal,\\*
škoda bude těch let,\\*
kdy jsem tě nelíbal,\\*
ještě slunci zamávám\\*
a potom líto přijde mi,\\*
že tě lásko nechávám\\*
samotnou tady na zemi.\\*
Že tě lásko nechávám\\*
na zemi.

Až zítra ráno v pět\\*
prádlo půjdeš prát\\*
a seno obracet\\*
já u  zdi budu stát,\\*
tak přilož na oheň\\*
a smutek v sobě skryj,\\*
prosím nezapomeň,\\*
nezapomeň a žij.\\*
Ná-na ná-na ná-na ná\\*
ná-na ná-na ná,\\*
na mě nezapomeň\\*
a žij.

\end{poem}

% TOP %
%\renewcommand*{\topname}{Anglické} % Name for the table of songs
%\maketop

\section{Slovenské}

Slovensko (anglicky Slotakia, maďarsky Kibaszott északi ország), dlouhý tvar
Slovenská republika, někdy též Severní Maďarsko, je malinkatá země ve tvaru
halušky, hraničící se Sovětským svazem. Žijí v ní (kromě Maďarů, Romů, Rusínů,
Poláků, Čechů a Hejslováků) také Slováci jako nejvýznamnější národnostní
menšina. 

Vysvětlení pojmenování Slovák a Hejslovák pochází od hanlivého oslovení, které
používali maďarští nájezdníci na porobené Slovany (Slovan $\rightarrow$ Slovák,
srv. Rakušan $\rightarrow$ Rakušák, Sudeťan $\rightarrow$ Sudeťák) a zvolací
částice Hej, kterou tito popoháněli své slovanské nevolníky do práce. 

Dle vlastních citací jsou nejstarším civilizovaným národem Evropy, a možná
i světa. Jsou prapředky všech známých starověkých civilizací, jako Etruskové,
Řekové, Římané, Egypťané, Indové, Číňané, Komanči a Brňáci. 

Hymna Nad Tatrou sa blýska svým meteorologickým zaměřením odkazuje na
slotistický původ. Zpívá se v licenci od Tatry Kopřivnice, která ji složila
jako reklamní popěvek oslavující dobré jízdní vlastnosti jejich vozů
v nepříznivém počasí. Textaři bohužel rytmicky nevycházel závěr Hejslováci
ožijú, a proto předponu Hej vynechal. 

V dobách Československa Slováci původně vylobbovali, aby se jejich hymna hrála
jako první, nicméně po protestech posluchačů ("tak přestaňte ladit a začněte
konečně hrát!") ji zakomponovali jako codu písně Kde domov můj.

Nejdůležitější fráze:
\begin{itemize}
	\item Mé vznášedlo je plné úhořů. – Moje vznášadlo je plné úhorov.
    \item Příliš žluťoučký kůň úpěl ďábelské ódy. – Kŕdeľ šťastných ďatľov učí
    pri ústí Váhu mĺkveho koňa obhrýzať kôru a žrať čerstvé mäso. 
\end{itemize}
    

\newpage
\thispagestyle{empty}

\begin{poem}{L. A. G. Song}{Horkýže Slíže}

\settowidth{\versewidth}{tak otvor branu lebo ju rozbijem!}

Poď sem, nech sa s tebou zblížim.\\*
Poď sem, ja ti neubližim.\\*
Poď sem, ja ťa nezbijem.\\*
I'm sorry, I'm a Lesbian.

Poď sem, som tvoj dvorny basnik.\\*
Mam energiu za dvanastich\\*
a k tomu lubrikačný gel.\\*
I'm sorry, I'm a really Gay.

Lesbian's and Gay's song.\\*
Lesbian's and Gay's sooooong.\\*
Lesbian's and Gay's song.\\*
Lesbian's and Gay's song.

Poď sem, pustim Iron Maiden.\\*
Poď sem, nalejem ti za jeden.\\*
Poď sem, s barskym nepijem.\\*
I'm sorry, I'm a Lesbian.

Poď sem, tu si ku mne hačaj.\\*
Keď som ťa pozval na rum a čaj.\\*
Stoj! Nechod nikam! Čo je? Hej!\\*
I'm sorry, I'm a really Gay.

Lesbian's and Gay's song.\\*
Lesbian's and Gay's sooooong.\\*
Lesbian's and Gay's song.\\*
Lesbian's and Gay's song.

Poď sem, spolu mame v plane.\\*
že ťa okupem vo fontane.\\*
tak otvor branu lebo ju rozbijem!\\*
I'm sorry baby, I'm a Lesbian.

Viem, neopakuj mi to stale\\*
ja som vyrastol na Death Metale,\\*
ja takymto veciam rozumiem.\\*
I'm sorry... ale veď ja viem!

Lesbian's and Gay song.\\*
Lesbian's and Gay sooooong.\\*
Lesbian's and Gay song.\\*
Lesbian's and Gay song.

\end{poem}

\section{Anglické}

Anglie, anglicky England, je jedna z historických zemí Spojených emirátů Velké
Británie a Severního Irska bez Skotska. Anglie přinesla pro lidstvo a světovou
kulturu mnoho pozitivního: 
\begin{itemize}
	\item Anglická kuchyně: Je druhá nejhorší kuchyně na světě, po kuchyni kmene
	Umburu v Papui
	\item Anglický humor: Podivná věta, které se Angličani smějí a nikdo neví,
	proč. 
	\item Anglické pivo: Je druhé nejhorší pivo na světě, po americkém. Vyznačuje
	se tím, že nemá žádnou pěnu. 
	\item Anglický politik: Neville Chamberlain svět naučil, jak úspěšně vyjednat
	Mnichovskou dohodu. 
	\item Anglický opilec: "Nazdar, chceš se prát, že jo?" "Ne, nechci."
	"Ale já myslím, že jo, takže tě musím praštit."
	\item Anglický plnokrevník: Teplokrevné plemeno koní vzniklé v Anglii křížením
	původních koní s Araby. 
	\item Anglický polokrevník: Teplokrevné plemeno koní, má jen polovinu
	vlastností a kvalit anglického plnokrevníka (tzv. dvounohý kůň).
	\item Beatles: Angličtí Brouci z Liverpoolu byli ve své době ve světě stejně
	populární, jako americký brouk v ČSSR. 
\end{itemize}

Důležitým uměleckým dílem, ve kterém se tato země zmiňuje, je film Svatba
Jiřího Káry z roku 2000. Tento film pojednává o svatbě bezdomovců a opilců na
jednom s pražských městských úřadů. V jedné pasáži hlavní postava filmu - Jiří
Kára - říká doslova "Anglie, Anglie, Anglie!" a následuje mohutné krknutí.
V době natočení šlo o jeden z prvních filmů zmiňující se o této zemi, a má za
následek výrazný přínos informací o tomto státě, jeho kultuře, zvycích
a historii českým občanům. 

\newpage
\thispagestyle{empty}

% \mainmatter

\begin{poem}{Blowin' in the Wind}{Bob Dylan}

\settowidth{\versewidth}{Yes, and how many times must the cannonballs fly}

How many roads must a man walk down\\*
Before you call him a man?\\*
Yes, and how many seas must a white dove sail\\*
Before she sleeps in the sand?\\*
Yes, and how many times must the cannonballs fly\\*
Before they're forever banned?\\*
The answer, my friend, is blowing in the wind\\*
The answer is blowing in the wind

Yes, how many years can a mountain exist\\*
Before it is washed to the sea?\\*
Yes, and how many years can some people exist\\*
Before they're allowed to be free?\\*
Yes, and how many times can a man turn his head\\*
And pretend that he just doesn't see?\\*
The answer, my friend, is blowing in the wind\\*
The answer is blowing in the wind

Yes, How many times must a man look up\\*
Before he can see the sky?\\*
Yes, and how many ears must one man have\\*
Before he can hear people cry?\\*
Yes, and how many deaths will it take till he knows\\*
That too many people have died?\\*
The answer, my friend, is blowing in the wind\\*
The answer is blowing in the wind

\end{poem}

\begin{poem}{Cocaine Blues}{Johnny Cash}

\settowidth{\versewidth}{He said come on you dirty heck into that district court.}

Early one mornin' while makin' the rounds,\\*
I took a shot of cocaine and I shot my woman down,\\*
I went right home and I went to bed,\\*
I stuck that lovin' .44 beneath my head.

Got up next mornin' and I grabbed that gun,\\*
Took a shot of cocaine and away I run,\\*
Made a good run but I ran too slow,\\*
They overtook me down in Juarez, Mexico.

Late in the hot joints takin' the pills,\\*
In walked the sheriff from Jericho Hill,\\*
He said Willy Lee your name is not Jack Brown,\\*
You're the dirty heck that shot your woman down.

Said yes, oh yes my name is Willy Lee,\\*
If you've got the warrant just a-read it to me,\\*
Shot her down because she made me sore,\\*
I thought I was her daddy but she had five more.

When I was arrested I was dressed in black,\\*
They put me on a train and they took me back,\\*
Had no friend for to go my bail,\\*
They slapped my dried up carcass in that county jail.

Early next mornin' bout a half past nine,\\*
I spied the sheriff coming down the line,\\*
Ah, and he coughed as he cleared his throat,\\*
He said come on you dirty heck into that district court.

Into the courtroom my trial began,\\*
Where I was handled by twelve honest men,\\*
Just before the jury started out,\\*
I saw the little judge commence to look about.

In about five minutes in walked the man,\\*
Holding the verdict in his right hand,\\*
The verdict read murder in the first degree,\\*
I hollered Lawdy Lawdy, have a mercy on me.

The judge he smiled as he picked up his pen,\\*
99 years in the Folsom pen,\\*
99 years underneath that ground,\\*
I can't forget the day I shot that bad bitch down.

Come on you've gotta listen unto me,\\*
Lay off that whiskey and let that cocaine be.

\end{poem}

\begin{poem}{Drunken Sailor}{lidová}

\settowidth{\versewidth}{Put him in the bed with the captains daughter,}

What shall we do with the drunken sailor?\\*
What shall we do with the drunken sailor?\\*
What shall we do with the drunken sailor?\\*
Early in the morning!

Way hey and up she rises,\\*
way hey and up she rises,\\*
way hey and up she rises,\\*
early in the morning!

Shave his belly with a rusty razor,\\*
shave his belly with a rusty razor,\\*
shave his belly with a rusty razor,\\*
early in the morning!

Way hey and up she rises,\\*
way hey and up she rises,\\*
way hey and up she rises,\\*
early in the morning!

Put him in a long boat till his sober,\\*
put him in a long boat till his sober,\\*
put him in a long boat till his sober,\\*
early in the morning!

Way hey and up she rises,\\*
way hey and up she rises,\\*
way hey and up she rises,\\*
early in the morning!

Stick him in a barrel with a hosepipe on him,\\*
stick him in a barrel with a hosepipe on him,\\*
stick him in a barrel with a hosepipe on him,\\*
early in the morning!

Way hey and up she rises,\\*
way hey and up she rises,\\*
way hey and up she rises,\\*
early in the morning!

Put him in the bed with the captain's daughter,\\*
put him in the bed with the captain's daughter,\\*
put him in the bed with the captain's daughter,\\*
early in the morning!

Way hey and up she rises,\\*
way hey and up she rises,\\*
way hey and up she rises,\\*
early in the morning!

That’s what we do with the drunken sailor,\\*
that’s what we do with the drunken sailor,\\*
that’s what we do with the drunken sailor,\\*
early in the morning!

Way hey and up she rises,\\*
way hey and up she rises,\\*
way hey and up she rises,\\*
early in the morning!

Way hey and up she rises,\\*
way hey and up she rises,\\*
way hey and up she rises,\\*
early in the morning!

\end{poem}

\begin{poem}{Hallelujah}{Leonard Cohen}

\settowidth{\versewidth}{Put him in the bed with the captains daughter,}

Now I've heard there was a secret chord,\\*
that David played, and it pleased the Lord,\\*
but you don't really care for music, do you?\\*
It goes like this, the fourth, the fifth,\\*
the minor fall, the major lift,\\*
the baffled king composing Hallelujah.

Hallelujah, Hallelujah,\\*
Hallelujah, Hallelujah.

Your faith was strong but you needed proof,\\*
you saw her bathing on the roof,\\*
her beauty and the moonlight overthrew you.\\*
She tied you to a kitchen chair,\\*
she broke your throne, and she cut your hair\\*
and from your lips she drew the Hallelujah.

Hallelujah, Hallelujah,\\*
Hallelujah, Hallelujah.

You say I took the name in vain,\\*
I don't even know the name,\\*
But if I did—well, really—what's it to you?\\*
There's a blaze of light in every word,\\*
It doesn't matter which you heard,\\*
the holy or the broken Hallelujah.

Hallelujah, Hallelujah,\\*
Hallelujah, Hallelujah.

I did my best, it wasn't much,\\*
I couldn't feel, so I tried to touch,,\\*
I've told the truth, I didn't come to fool you.\\*
And even though it all went wrong,\\*
I'll stand before the Lord of Song,\\*
with nothing on my tongue but Hallelujah.

Hallelujah, Hallelujah,\\*
Hallelujah, Hallelujah.

Hallelujah, Hallelujah,\\*
Hallelujah, Hallelujah.

Hallelujah, Hallelujah,\\*
Hallelujah, Hallelujah.

Hallelujah, Hallelujah,\\*
Hallelujah, Hallelujah.

\end{poem}

\begin{poem}{Jack of Hands}{Edward Ka-Spel}

\settowidth{\versewidth}{He’s free to tread and lift the sheets}

Charity begins at homes\\*
For those too sick, those indisposed \\*
Footsteps in the corridor \\*
Little Johnny’s playing dead

They gave the Jack of Hands the key \\*
He’s free to tread and lift the sheets \\*
He’ll be dressed in white they said

But I don’t believe in angels

When the good son rises \\*
When we hear the birds \\*
Little Johnny's curled up and crying \\*
Alas a touch disturbed

The Jack of Hands is far away \\*
He’s opening a school \\*
They’ll offer him a chariot \\*
But Jack prefers to walk 

Dangerous

\vfill\eject

\begin{altverse}
Hush now \\*
Hush now \\*
Jack is coming \\*
Jack is coming \\*
Shush now \\*
Shush now \\*
Like you’re sleeping \\*
Like you’re sleeping\\*
\end{altverse}
He’s carrying his special box \\*
No one ever asks what’s in it \\*
No more suffer little children \\*
\begin{altverse}
Jack’ll fix it \\*
Jack’ll fix it 
\end{altverse}


It’s Jack the national treasure \\*
Jack’s beating on his chest \\*
He’s flying off to Africa \\*
Cause Africa’s the best 

They’ll cordon off the airport \\*
Because the crowd is vast \\*
And Jack will play the trinity \\*
He’ll be home at last 

Still I don’t believe in angels

\end{poem}

\begin{poem}{Old Maui}{lidová}

\settowidth{\versewidth}{Cause we're homeward bound from the Arctic ground}

'Tis a damn tough life full of toil and strife\\*
We whalermen undergo\\*
And we don't give a damn when the day is done\\*
How hard the winds did blow\\*
Cause we're homeward bound from the Arctic ground\\*
With a good ship, taut and free\\*
And we don't give a damn when we drink our rum\\*
With the girls of Old Maui

Rolling down to Old Maui, me boys\\*
Rolling down to Old Maui\\*
We're homeward bound from the Arctic ground\\*
Rolling down to Old Maui

Once more we sail with a northerly gale\\*
Towards our island home\\*
Our mainmast sprung, our whaling done\\*
And we ain't got far to roam\\*
Six hellish months we passed away\\*
On the cold Kamchatka Sea\\*
But now we're bound from the Arctic ground\\*
Rolling down to Old Maui

Rolling down to Old Maui, me boys\\*
Rolling down to Old Maui\\*
We're homeward bound from the Arctic ground\\*
Rolling down to Old Maui

Once more we sail with a northerly gale\\*
Through the ice and wind and rain\\*
Them coconut fronds, them tropical lands\\*
We soon shall see again\\*
Even now their big brown eyes look out\\*
Hoping some fine day to see\\*
Our baggy sails running 'fore the gale\\*
Rolling down to Old Maui

Rolling down to Old Maui, me boys\\*
Rolling down to Old Maui\\*
We're homeward bound from the Arctic ground\\*
Rolling down to Old Maui

How soft the breeze through the island trees\\*
Now the ice is far astern\\*
Them native maids, them tropical glades\\*
Is a-waiting our return\\*
I will rant and roar and row to shore\\*
And paint them beaches red\\*
Waking in the arms of a Wahine maid\\*
With a big fat aching head

Rolling down to Old Maui, me boys\\*
Rolling down to Old Maui\\*
We're homeward bound from the Arctic ground\\*
Rolling down to Old Maui

Rolling down to Old Maui, me boys\\*
Rolling down to Old Maui\\*
We're homeward bound from the Arctic ground\\*
Rolling down to Old Maui

\end{poem}

\begin{poem}{Over the Rainbow}{Judy Garland}

\settowidth{\versewidth}{Wake up where the clouds are far behind me}

Somewhere over the rainbow\\*
Way up high\\*
And the dreams that you dream of\\*
Once in a lullaby

Somewhere over the rainbow\\*
Bluebirds fly\\*
And the dreams that you dream of\\*
Dreams really do come true

Someday, I wish upon a star\\*
Wake up where the clouds are far behind me\\*
Where trouble melts like lemon drops\\*
High above the chimney top\\*
That's where you'll find me

Somewhere over the rainbow\\*
Bluebirds fly\\*
And the dreams that you dare to\\*
Oh why, oh why can't I?

\end{poem}

\begin{poem}{Ring of Fire}{Johnny Cash}

\settowidth{\versewidth}{I fell into a burnin' ring of fire}

\incipit{Love is a burnin' thing}\\*
And it makes a firery ring\\*
Bound by wild desire\\*
I fell into a ring of fire.

/:I fell into a burnin' ring of fire\\*
I went down, down, down\\*
And the flames went higher\\*
And it burns, burns, burns\\*
The ring of fire, the ring of fire.:/

The taste of love is sweet,\\*
when hearts like ours meet.\\*
I fell for you like a child,\\*
oh, but the fire went wild.

/:I fell into a burnin' ring of fire\\*
I went down, down, down\\*
And the flames went higher\\*
And it burns, burns, burns\\*
The ring of fire, the ring of fire.:/

And it burns, burns, burns.\\*
The ring of fire,\\*
the ring of fire,\\*
the ring of fire,\\*
the ring of fire,\\*
the ring of fire.

\end{poem}

\begin{poem}{Rocky Road to Dublin}{lidová}

\settowidth{\versewidth}{And I have frightened all the dogs on the rocky road
to Dublin,}

While in the merry month of May from me home I started\\*
Left the girls of Tuam nearly broken-hearted\\*
Saluted father dear, kissed me darlin' Mother\\*
Drank a pint of beer me grief and tears to smother\\*
Then off to reap the corn, and leave where I was born\\*
Cut a stout blackthorn to banish ghost and goblin,\\*
In a bran' new pair of brogues I rattled o'er the bogs\\*
And frightened all the dogs on the rocky road to Dublin,\\*
One, two, three, four, five. 

Ref.:\\*
Hunt the hare and turn her\\*
Down the rocky road, and all the ways to Dublin\\*
Whack fol-lol-de-ra.

In Mullingar that night I rested limbs so weary,\\*
Started by daylight next morning light and airy,\\*
Took a drop of the pure, to keep my heart from \mbox{sinking},\\*
That's the Paddy's cure, whene'er he's on drinking,\\*
To see the lasses smile, laughing all the while,\\*
At my curious style, 'twould set your heart a-bubblin'\\*
They ax'd if I was hired, the wages I required,\\*
Till I was almost tired of the rocky road to Dublin.\\*
One, two, three, four, five. 

Ref.

In Dublin next arrived, I thought it such a pity,\\*
To be so soon deprived a view of that fine city,\\*
Then I took a stroll out among the quality,\\*
My bundle it was stole in a neat locality;\\*
Something crossed my mind, then I looked behind,\\*
No bundle could I find upon me stick a-wobblin',\\*
Enquiring for the rogue, they said my Connaught brogue\\*
Wasn't much in vogue on the rocky road to Dublin.\\*
One, two, three, four, five. 

Ref.

From there I got away my spirits never failing,\\*
Landed on the quay as the ship was sailing,\\*
Captain at me roared, said that no room had he,\\*
When I jumped aboard, a cabin found for Paddy\\*
Down among the pigs, I played some funny rigs\\*
Danced some hearty jigs, the water round me \mbox{bubblin}\\*
When off to Holyhead I wished myself was dead,\\*
Or better far, instead, on the rocky road to Dublin.\\*
One, two, three, four, five. 

Ref.

The boys of Liverpool, when we safely landed,\\*
Called myself a fool, I could no longer stand it;\\*
Blood began to boil, temper I was losin'\\*
Poor old Erin's isle they began abusin'\\*
"Hurrah my soul!" sez I, my shillelagh I let fly,\\*
Some Galway boys were by, saw I was a hobble in,\\*
Then with a loud Hurrah, they joined in the affray,\\*
We quickly cleared the way, for the rocky road to Dublin.\\*
One, two, three, four, five. 

Ref.

\end{poem}

\begin{poem}{Roll the Woodpile Down}{lidová}

\settowidth{\versewidth}{Rollin', rollin', rollin' the whole world 'round}

Away down South where the cocks do crow\\*
Way down in Florida\\*
Them girls all dance to the old banjo\\*
And we'll roll the woodpile down

Rollin', rollin', rollin' the whole world 'round\\*
That brown gal of mine's on the Georgia line\\*
And we'll roll the woodpile down

Oh, what can you do in Tampa bay?\\*
Way down in Florida\\*
But give them yellow girls all your pay\\*
And we'll roll the woodpile down

Rollin', rollin', rollin' the whole world 'round\\*
That brown gal of mine's on the Georgia line\\*
And we'll roll the woodpile down

Them Cardiff girls ain't got no frills\\*
Way down in Florida\\*
They're skinny and tight as catfish gills\\*
And we'll roll the woodpile down

Rollin', rollin', rollin' the whole world 'round\\*
That brown gal of mine's on the Georgia line\\*
And we'll roll the woodpile down

Oh, why do them little girls love me so?\\*
Way down in Florida\\*
Because I don't tell all I know\\*
And we'll roll the woodpile down

Rollin', rollin', rollin' the whole world 'round\\*
That brown gal of mine's on the Georgia line\\*
And we'll roll the woodpile down

Oh, one more pull and that will do\\*
Way down in Florida\\*
For we're the boys to kick her through\\*
And we'll roll the woodpile down

Rollin', rollin', rollin' the whole world 'round\\*
That brown gal of mine's on the Georgia line\\*
And we'll roll the woodpile down\\*
That brown gal of mine's on the Georgia line\\*
And we'll roll the woodpile down

\end{poem}

\begin{poem}{The Times They Are A-Changin'}{Bob Dylan}

\settowidth{\versewidth}{Please get outta' the new one if you can’t lend your hand}

Come gather 'round people\\*
Wherever you roam\\*
And admit that the waters\\*
Around you have grown\\*
And accept it that soon\\*
You’ll be drenched to the bone\\*
If your time to you is worth saving\\*
Then you better start swimmin' or you’ll sink like a~stone\\*
For the times they are a-changin'.

Come writers and critics\\*
Who prophesize with your pen\\*
And keep your eyes wide\\*
The chance won’t come again\\*
And don’t speak too soon\\*
For the wheel’s still in spin\\*
And there’s no tellin' who that it’s namin'\\*
For the loser now will be later to win\\*
For the times they are a-changin'.

\vfill\eject

Come mothers and fathers\\*
Throughout the land\\*
And don’t criticize\\*
What you can’t understand\\*
Your sons and your daughters\\*
Are beyond your command\\*
Your old road is rapidly agein'\\*
Please get outta' the new one if you can’t lend your hand\\*
For the times they are a-changin'.

Come senators, congressmen\\*
Please heed the call\\*
Don’t stand in the doorway\\*
Don’t block up the hall\\*
For he that gets hurt\\*
Will be he who has stalled\\*
The battle outside ragin'\\*
Will soon shake your windows and rattle your walls\\*
For the times they are a-changin'.

The line it is drawn\\*
The curse it is cast\\*
The slow one now\\*
Will later be fast\\*
As the present now\\*
Will later be past\\*
The order is rapidly fading\\*
And the first one now will later be last\\*
For the times they are a-changin'.

\end{poem}

\begin{poem}{Toxicity}{System of a Down}

\settowidth{\versewidth}{Flashlight reveries caught in the headlights of a truck}

Conversion, software version 7.0\\*
Looking at life through the eyes of a tired hub\\*
Eating seeds as a pastime activity\\*
The toxicity of our city, our city

Ref.:\\*
You, what do you own the world?\\*
How do you own disorder, disorder\\*
Now somewhere between the sacred silence\\*
Sacred silence and sleep\\*
Somewhere, between the sacred silence and sleep\\*
Disorder, disorder, disorder

More wood for the fires, loud neighbors\\*
Flashlight reveries caught in the headlights of a truck\\*
Eating seeds as a pastime activity\\*
The toxicity of our city, of our city

Ref.

Ref.

When I became the sun\\*
I shone life into the man's hearts\\*
When I became the sun\\*
I shone life into the man's hearts

\end{poem}

\section{Italské}

Itálie je nevyspělou zemí divokých Italů. Nachází se v jižní Evropě a je
obklopena ze všech stran mořem (kromě severu). Je tam dostatek slunce, vína,
italské zmrzliny, italské mafie a velký nadbytek pláží. Proto se v italském
parlamentu před několika lety uvažovalo o odprodeji několika italských pláží
České republice (na žádost Jiřího Paroubka). Měly být přemístěny do jižních
Čech. Tento obchod se však nakonec neuskutečnil protože se zjistilo, že by bylo
nutno odprodat i kus Jadranského moře, a to si nárokuje též Chorvatsko. 

Itálie ráda vstupuje a vystupuje z mezinárodních organizací, spolků,
spojenectví. Dnes je v OSN, FIFA, EU, G7, G8, SPQR nebo NATO, účast v eurozóně
zpochybnila a účast v Evropské unii také, což o 6 minut později vzala zpět.
Patřila k Ose, později odešla. V letech 974 až 1456 součástí Svaté říše římské
s výjimkou let 1022 až 1026 a 1170 až 1182 (v letech 1182 až 1186 tak napůl).
V roce 1456 odešla, ještě dříve se vrátila a zůstala až do konce, když se díky
Napoleonovi rozpadla. 

Itálie se skládá z nohy, která nakopává Sicílii, což má svůj historický původ
v římských dobách, kdy Římané válčili se Sicílií. Další ostrov, Sardinie, byl
již Římany definitivně odkopnut. Dříve jim náležela též Libye, Itálie
potřebovala pevnou půdu pod nohama, ale přišla o ni. 

Itálie dala světu hudbu, bez níž se neobešla žádná normalizační estráda. A to
přestože šlo povětšinou o obšlehlé verze českých autorů, jako byl Jaroslav
Uhlíř, Těžkej Pokondr či Semtex. Samotná hudba se přitom vyznačuje kreativním
minimalismem, kdy s minimem prostředků autoři vykouzlí hromady skladeb. Většina
písní má harmonii na akordy C - Ami - Dmi - G a slova jsou plusmínus stejná.
Písně se však dají dobře odlišit podle rytmu a přízvuku v refrénu
(ámore / amóre / amoré).

Italský minimalismus se projevuje i ve filmu. Jen tak si lze
vysvětlit, že byli schopni vytvořit sérii úspěšných filmových
komedií na základě toho, že jeden chlápek dá facku druhému. 

\newpage
\thispagestyle{empty}

% \mainmatter

\begin{poem}{Bella Ciao}{lidová}

\settowidth{\versewidth}{O bella ciao, bella ciao, bella ciao ciao ciao}

\incipit{Una mattina mi son svegliato}\\*
O bella ciao, bella ciao, bella ciao ciao ciao\\*
Una mattina mi son svegliato\\*
Eo ho trovato l'invasor.

O partigiano porta mi via\\*
O bella ciao, bella ciao, bella ciao ciao ciao\\*
O partigiano porta mi via\\*
Che mi sento di morir.

E se io muoio da partigiano\\*
O bella ciao, bella ciao, bella ciao ciao ciao\\*
E se io muoio da partigiano\\*
Tu mi devi seppellir.

Mi seppellire lassù in montagna\\*
O bella ciao, bella ciao, bella ciao ciao ciao\\*
Mi seppellire lassù in montagna\\*
Sotto l'ombra di un bel fiore.

E le genti che passeranno\\*
O bella ciao, bella ciao, bella ciao ciao ciao\\*
E le genti che passeranno\\*
Mi diranno: "Che bel fior".

È questo il fiore del partigiano\\*
O bella ciao, bella ciao, bella ciao ciao ciao\\*
È questo il fiore del partigiano\\*
Morto per la libertà.

\end{poem}

\end{document}\grid
